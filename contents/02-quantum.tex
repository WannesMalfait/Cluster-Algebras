
\chapter{Quantum cluster algebras}
All the algebras that we have discussed thus far have been commutative. Indeed, a
cluster algebra is by construction commutative. There are, however, many interesting
algebras arising from natural examples that are not commutative. Of these, a broad
class is given by deformations of coordinate algebras of algebraic varieties. To also
be able to study these objects through cluster algebras, we will need a noncommutative
version of a cluster algebra. The first part of this chapter will be dedicated to
describing how one can generalize cluster algebras to the noncommutative setting,
obtaining so-called quantum cluster algebras.

We will then compare the Grassmannians that we saw in the setting of cluster algebras,
to quantum Grassmannians. This will be used to provide a better intuition for the
differences between the quantum and non-quantum setting. In the next chapter we will
look at one specific class of noncommutative algebras, the symmetric CGL extensions.
These always admit a quantum cluster algebra structure, which can be written down
explicitly. We will follow the work of \cite{GoodearlYakimov2017QCA} while working out
some examples to get a better feel for the material.

Let us now first fix some notation. In what follows, $\bbK$ will be a field and $N \in
	\bbZ_{> 0}$ an integer. We denote by $\bbK^\times$ the multiplicative group of $\bbK$.
It will be useful to consider the following multiplicative version of matrix
multiplication. For $A = (a_{ij}) \in M_{m \times n}(\bbK^\times), B = (b_{ij}) \in
	M_{s \times m}(\bbZ)$ and $C = (c_{ij}) \in M_{n \times r} (\bbZ)$, we define
$\prescript{B}{}{}A^C \in M_{s \times r}(\bbK^\times)$ as the matrix with entries
\begin{equation*}
	(\prescript{B}{}{A}^C)_{ij} = \prod_{k, l} a_{kl}^{b_{ik}c_{lj}},
\end{equation*}
%
with special cases
\begin{equation*}
	(\prescript{B}{}{A})_{ij} = \prod_{k} a_{kj}^{b_{ik}}, \text{ and }	(A^C)_{ij} = \prod_{k} a_{ik}^{c_{kj}}.
\end{equation*}
%
This product is also associative in the sense that $\prescript{B_2 B_1}{}{A}^{C_1 C_2}
	= \prescript{B_2}{}{}(\prescript{B_1}{}{}A^{C_1})^{C_2}$.

In most cases the exponent will be a column vector. Whenever we write $f \in \bbZ^N$ we
will view $f$ as a column vector, i.e., an $N \times 1$-matrix over the integers. The
standard basis vectors of $\bbZ^N$ will be denoted by $e_1, \dots, e_N$.

For $m, n \in \bbZ$, we will write $[m, n]$ to denote the set $\{m, m+1, \dots, n-1,
	n\} \in \bbZ$. In particular, $[m, n] = \emptyset$ if $m > n$.

\section{Ore localizations and tori}

Recall that the cluster variables were initially defined as elements of the function
field $\bbK(x_1, \dots, x_N)$. It is not so trivial to come up with a good
noncommutative analog of this function field. However, due to the Laurent phenomenon
(\cref{thm:laurent_phenomenon}), it suffices to consider the algebra $\bbK[x_1^{\pm 1},
	\dots, x_N^{\pm 1}]$. Note that this is the coordinate algebra of the
\emph{$\bbK$-torus}\index{torus}, $(\bbK^\times)^N$. We can then apply a minimal
deformation to this coordinate algebra to make it noncommutative. This yields a
so-called \emph{quantum torus}\index{torus!quantum}. These quantum tori will play a
central role in what follows.

Let $\bbq \in M_N(\bbK^\times)$ be a \emph{multiplicatively
	skew-symmetric}\index{skew-symmetric!multiplicatively} matrix, i.e., $r_{ij} =
	r_{ji}\inv$ for all $i,j \in [1, N]$. Deforming the affine space coordinate algebra
$\bbK[Y_1, \dots, Y_N]$ by $\bbq$, we obtain
\begin{equation*}
	\mcA_\bbq = \frac{\bbK \langle Y_1 , \dots, Y_N\rangle}{(Y_i Y_j = q_{ij}Y_j Y_i)},
\end{equation*}
the \emph{quantum affine space algebra}\index{algebra!quantum affine space}. We then write $\mcT_\bbq$ for the \emph{quantum torus}
\begin{equation}\label{eq:quantum_torus}
	\mcT_\bbq = \frac{\bbK \langle Y_1^{\pm 1} , \dots, Y_N^{\pm 1}\rangle}{(Y_i Y_j = q_{ij}Y_j Y_i)}.
\end{equation}
%
It has a $\bbK$-basis consisting of elements of the form $Y^f = Y_1^{f_1} \cdots
	Y_N^{f_N}$ for $f \in \bbZ^N$. We will refer to the property that $X Y = qY X$ for some
$q \in \bbK^\times$ by saying that $Y_i$ and $Y_j$
\emph{quasi-commute}\index{quasi-commuting}.
\begin{remark}

	In most examples, $\bbq$ will be of the form $(q^{\Lambda_{ij}})$ for some
	skew-symmetric matrix $\Lambda \in M_N(\bbZ)$ and $q\in \bbK^\times$. We will refer to
	$\mcT_\bbq$ as a \emph{uniparameter}\index{uniparameter} quantum torus in this case.
\end{remark}
%

Even though a quantum version of the Laurent phenomenon will hold
(\cref{thm:quantum_laurent}), we will still need to be able to invert arbitrary
(non-zero) elements in $\mcA_\bbq$ in order to define a quantum cluster algebra. This
requires localization in the noncommutative setting, also known as \emph{Ore
	localization}\index{Ore localization}.

The idea is that we want to define a new ring $R[E\inv] = \{re\inv \mid r \in R,\,e \in
	E\}$ for some subset $E \subseteq R$ of a (noncommutative) ring $R$. In order for this
to make sense, $E$ should be a \emph{multiplicative subset}\index{multiplicative
	subset}, i.e., $1 \in E$ and $ef \in E$ for all $e,f \in E$. Furthermore, $E$ should
consist of \emph{regular elements}\index{element!regular}, i.e., nonzero elements that
are neither left nor right zero-divisors. This alone is not enough: additionally, if
$e, f \in E$, then for any $x, y \in R$ we should have $xe\inv yf\inv \in R[E\inv]$.
This would be satisfied if we could write $e\inv y = y' (e')\inv$ for some $y' \in R$
and $e' \in E$. That condition is equivalent to $y e' = e y'$ for some $y' \in R$ and
$e' \in E$, which in turn is equivalent to $y E \cap e R \neq \emptyset$ for all $y \in
	R$ and $e \in E$. Putting this together, we find the following definition.
\begin{definition}
	A \emph{right Ore set}\index{Ore set} $E
		\subseteq R$ is a multiplicative subset of regular elements such that $x E \cap e R
		\neq \emptyset$ for all $x \in R$ and $e \in E$. One defines a left Ore set
	analogously. If $E$ is both a left and a right Ore set, we will just say that $E$ is an Ore set.
\end{definition}

Let $E \subseteq R$ be a multiplicative subset of regular elements. We say that a ring
$S \supseteq R$ is a \emph{right ring of fractions for $R$ with respect to
	$E$}\index{ring!of fractions} if
\begin{enumerate}
	\item Every element of $E$ is invertible in $S$.
	\item Every element of $S$ can be expressed in the form $re\inv$ for some $r \in R$ and $e
		      \in E$.
\end{enumerate}
The following theorem justifies the definition of an Ore set.
\begin{theorem}\cite[Theorem 6.2]{GoodearlWarfield2004NoncommutativeNR}
	Let $E \subseteq R$ be a multiplicative subset of regular elements of a ring $R$. Then $R$ admits a right ring of fractions with respect to $E$ if and only if $E$ is a right Ore set.
\end{theorem}
%
Ore localization shares many common properties with localization in the commutative
setting. The following lemma shows, for example, that we can always find common
denominators. On the other hand, one should also be careful. Testing equality of two
fractions is a bit more involved.
\begin{lemma}[\protect{\cite[Lemma 6.1]{GoodearlWarfield2004NoncommutativeNR}}]\label{lem:ore_set_properties}
	Let $R$ be a ring and $E$ a right Ore set. Let $S$ be a right ring of fractions with respect to $E$.
	\begin{enumerate}
		\item Given fractions $s_1, \dots, s_n \in S$, we can find a common denominator. Thus, there
		      exist $r_1, \dots, r_n \in R$ and $e \in E$ such that $s_i = r_i e\inv$ for all $i \in
			      [1, n]$.
		\item For $x, y \in R$ and $e, f \in E$, we have $x e \inv = y f \inv$ if and only if there
		      exist $a,b \in R$ such that $xa = yb$ and $ea = fb \in E$.
	\end{enumerate}
\end{lemma}

As an example, note that $\mcT_\bbq$ is a right (and left) ring of fractions of
$\mcA_\bbq$ with respect to the multiplicative subset, $E$, generated by $\bbK^\times$
and $Y_1, \dots, Y_N$. Indeed, because the $Y_i$ quasi-commute and generate $\mcA_\bbq$
(as an algebra), we have $e \mcA_\bbq = \mcA_\bbq e$ for all $e \in E$. Furthermore,
all the nonzero elements in $\mcA_\bbq$ are regular. Thus, $E$ is an Ore set.

When $R$ is a (noncommutative) domain, the set, $R \setminus \{0\}$, is a
multiplicative subset of $R$ consisting of regular elements. When this is also an Ore
set, we write $\Fract(R)$\index{fract@$\Fract(R)$} for the corresponding ring of
fractions. For example, the set $\mcT_\bbq \setminus \{0\}$ forms an Ore set by a
similar argument as above. The quantum cluster variables will live in the division
algebra $\Fract(\mcT_\bbq)$.

\section{Toric frames and quantum mutations}

We now proceed with the quest towards quantum cluster algebras. Recall that a cluster
algebra is determined by an initial seed, consisting of a matrix and the initial
cluster variables. The mutation is completely determined by the matrix. In the
noncommutative setting, there is one additional thing to keep track of, namely the
multiplicatively skew-symmetric matrix $\bbq \in M_N(\bbK^\times)$. It turns out that
some sort of compatibility will be needed between the mutation matrix, $\tB$, and the
matrix $\bbq$.

Fix a multiplicatively skew-symmetric matrix $\bbq \in M_N (\bbK^\times)$, and a
mutation matrix $\tB \in M_{N \times \ex}(\bbZ)$. As in the classical setting, $\ex
	\subseteq [1, N]$ and the \emph{principal part}\index{principal part} of $\tB$ is the
submatrix $B \in M_{\ex \times \ex}(\bbZ)$ indexed by the elements of $\ex$. Let
$\mcT_\bbq$ be a quantum torus as in \cref{eq:quantum_torus}.

We now explore a naive way of generalizing the exchange relation
(\cref{eq:exchange_relation_quiver} \textbf{TODO: change to matrix version once added
	in chapter 1}) to the noncommutative setting. This will give us the intuition needed to
arrive at the right definition. Take $k \in \ex$ and write $b^k$ for the $k$-th column
of $\tB$. We will denote with $[b^k]_{+}$ and $[b^k]_{-}$ the positive and negative
parts of $b^k \in \bbZ^N$. Thus,
\begin{align*}
	[b^k]_{+} & = \sum_{i=1}^N \max\{0, b_{ik}\}e_i = \sum_{b_{ik} > 0} b_{ik}e_i. \\
	\shortintertext{and}
	[b^k]_{-} & = \sum_{i=1}^N \min\{0, b_{ik}\}e_i = \sum_{b_{ik} < 0} b_{ik}e_i.
\end{align*}
We then define $Y'_k \in \Fract(\mcT_\bbq)$ as
\begin{equation}\label{eq:quantum_exchange_naive}
	Y_k' = Y_k\inv \left(Y^{[b^k]_{+}} + Y^{-[b^k]_{-}}\right) = Y_k\inv \left(\prod_{b_{ik} > 0}Y_i^{b_{ik}} + \prod_{b_{ik} < 0} Y_i^{-b_{ik}}\right).
\end{equation}
%
By convention, we take all the products to be in increasing index\footnote{If this
	feels arbitrary, that is because it is. This will be addressed by switching over to
	toric frames.}, i.e., $Y^f = Y_1^{f_1}Y_2^{f_2} \cdots Y_N^{f_N}$ for $f \in \bbZ^N$.
For this definition to work, the variables $Y_1, \dots, Y_{k-1}, Y'_k, Y_{k+1}, \dots,
	Y_N$ should now be contained in another quantum torus $\mcT_{\bbq'}$. Thus, for each $j
	\in [1, N], j \neq k$, we need $Y'_k Y_j = q'_{kj} Y_j Y'_k$ for some $q'_{kj} \in
	\bbK^\times$. We have
\begin{align*}
	Y'_k Y_j
	 & = Y_k\inv \left(\prod_{b_{ik} > 0} Y_i^{b_{ik}} + \prod_{b_{ik} < 0}Y_i^{-b_{ik}}\right) Y_j                                             \\
	 & = Y_k\inv \left(\prod_{b_{ik} > 0} Y_i^{b_{ik}}Y_j + \prod_{b_{ik} < 0}Y_i^{-b_{ik}}Y_j\right)                                           \\
	 & = Y_k\inv \left(Y_j \prod_{b_{ik} > 0} q_{ij}^{b_{ik}} Y_i^{b_{ik}} + Y_j \prod_{b_{ik} < 0}q_{ij}^{-b_{ik}}Y_i^{-b_{ik}}\right)         \\
	 & = q_{kj}\inv Y_j Y_k\inv \left(\prod_{b_{ik} > 0} q_{ij}^{b_{ik}} Y_i^{b_{ik}} + \prod_{b_{ik} < 0}q_{ij}^{-b_{ik}}Y_i^{-b_{ik}}\right).
\end{align*}
%
It follows that we must have
\begin{equation}\label{eq:b_plus_is_b_minus}
	\prod_{b_{ik} > 0}q_{ij}^{b_{ik}} = \prod_{b_{ik} < 0}q_{ij}^{ - b_{ik}},
\end{equation}
which is equivalent to
\begin{equation}\label{eq:compatibility_q_b}
	\prod_{i=1}^N q_{ij}^{b_{ik}} = 1.
\end{equation}
%
Under this assumption, we then find
\begin{equation}\label{eq:mutated_q_kj}
	q'_{kj} = q_{kj}\inv \prod_{b_{ik} > 0}q_{ij}^{b_{ik}} = q_{kj}\inv \prod_{b_{ik} < 0}q_{ij}^{-b_{ik}}, \text{ for all } k \in \ex,\, j\in [1, N], j\neq k.
\end{equation}
%
This completely determines the new matrix $\bbq' = (q'_{ij})$, as we have $q'_{ij} =
	q_{ij}$ for $i,j \neq k$. We now introduce some notation to be able to write the above
formulas in a slightly more compact way.

To a multiplicatively skew-symmetric matrix $\bbq \in M_N(\bbK^\times)$ we associate a
\emph{skew-symmetric bicharacter}\index{skew-symmetric!bicharacter}
\begin{equation*}
	\Omega_{\bbq} \colon \bbZ^N \times \bbZ^N \to \bbK^\times \colon
	\Omega_{\bbq}(f, g) = \prescript{f^T}{}{}\bbq^g.
\end{equation*}
%
It is a bicharacter in the sense that each of the components is a character $\bbZ^N \to
	\bbK^\times$. The function $\Omega_\bbq$ satisfies (or is determined by) the following
properties
\begin{align}
	 & \Omega_\bbq (e_i, e_j) = q_{ij}, \quad \forall i, j \in [1, N],\label{eq:Omega_of_basis}                             \\
	 & \Omega_\bbq (f, g) = \Omega_\bbq (g, f)\inv,\quad \forall f, g \in \bbZ^N,\label{eq:Omega_skew_symmetric}            \\
	 & \Omega_\bbq (f + g, h) = \Omega_\bbq(f, h)\Omega_\bbq(g, h), \quad \forall f, g, h\in \bbZ^N.\label{eq:Omega_of_sum}
\end{align}
%
\Cref{eq:compatibility_q_b} can now be rewritten as $\Omega_\bbq(b^k, e_j) = 1$. This leads to the following definition.
\begin{definition}\label{def:compatible_pair}
	Let $\tB \in M_{N \times \ex}(\bbZ)$ be an integer matrix, and $\bbq \in M_N(\bbK^\times)$ be a multiplicatively skew-symmetric matrix.	Let $\wt{\bbt}\in M_{\ex \times N}(\bbZ)$ be the matrix with entries
	\begin{equation*}
		t_{kj} = \Omega_\bbq(b^k, e_j) = \prod_{i=1}^N q_{ij}^{b_{ik}}
	\end{equation*}
	for $k \in \ex$ and $j \in [1, N]$. We say that the pair $(\bbq, \tB)$ is \emph{compatible}\index{compatible pair} if $t_{kj} = 1$ for all $k \in \ex$ and $j \in [1, N], j\neq k$ and  $t_{kk}$ is not a root of unity, for all $k\in \ex$.
\end{definition}
\begin{remark}
	The condition that $t_{kk}$ is not a root of unity is not strictly necessary to ensure that $Y'_k$ quasi-commutes with the $Y_j, j\neq k$. It is more of a technical condition that prevents certain degenerate cases, and has some useful consequences (cf. \cref{lem:principal_part_skew_symmetrizable}).
\end{remark}
%

In order for $\tB$ to be a mutation matrix, the principal part of $\tB$ should be
skew-symmetrizable. Interestingly, this is almost automatic for compatible pairs.
\begin{lemma}\label{lem:principal_part_skew_symmetrizable}
	Let $(\bbq, \tB)$ be a compatible pair. Assume that there exist $d_k \in \bbZ_{>0}$ for $k \in \ex$ such that
	\begin{equation*}
		t_{kk}^{d_j} = t_{jj}^{d_k}, \text{ for all } j,k \in \ex.
	\end{equation*}
	%
	Then the principal part of $\tB$ is skew-symmetrizable by the $d_k, k\in \ex$.
\end{lemma}
\begin{proof}

	For $j,k \in \ex$ we have
	\begin{equation*}
		t_{kk}^{b_{kj}} = \prod_{i=1}^N t_{ki}^{b_{ij}} = \prod_{i=1}^N\prod_{l=1}^N q_{li}^{b_{lk}b_{ij}} =
		\prod_{i=1}^N\prod_{l=1}^N q_{il}^{-b_{lk}b_{ij}} = \prod_{l=1}^N t_{jl}^{-b_{lk}} = t_{jj}^{-b_{jk}}.
	\end{equation*}
	Thus,
	\begin{equation*}
		t_{jj}^{b_{kj} d_k} = t_{kk}^{ b_{kj}d_j} = t_{jj}^{- b_{jk} d_j},
	\end{equation*}
	from which we obtain $b_{kj} d_j = -b_{jk} d_j$ for all $j, k \in \ex$ due to the fact that $t_{ii}$ is not a root of unity for all $i \in \ex$.
\end{proof}

We now look at the mutation of the compatible pairs. Recall that the mutation of $\tB$
in direction $k$ was given by the formula
\begin{equation*}
	b'_{ij} = \begin{dcases*}
		-b_{ij},                                            & if $i = k$ or $j = k$ \\
		b_{ij} + \frac{|b_{ik}|b_{kj} + b_{ik}|b_{kj|}}{2}, & otherwise.
	\end{dcases*}
\end{equation*}
%
On the other hand, if $(\bbq, \tB)$ is a compatible pair, we know that
\cref{eq:mutated_q_kj} holds. Let $\epsilon = \pm 1$ and $E_{\epsilon} \in M_N(\bbZ)$
be the matrix with columns $(e_1, \dots, e_{k-1}, [-\epsilon b^k]_{+} - e_k, e_{k+1},
	\dots, e_N)$. We then define
\begin{equation}\label{eq:def_mutation_q}
	\mu_k(\bbq) = \bbq' = \prescript{E_{\epsilon}^T}{}{}\bbq^{E_\epsilon}.
\end{equation}
%
The choice of the sign $\epsilon = \pm 1$ corresponds precisely to choosing the first
or second inequality in \cref{eq:mutated_q_kj}. Because of the compatibility condition,
the definition of $\bbq'$ is independent of the choice of sign (cf.
\cref{eq:b_plus_is_b_minus}). We then define the \emph{mutation in direction $k \in
		\ex$}\index{mutation!of compatible pairs} of the compatible pair $(\bbq, \tB)$ to be
the pair $(\mu_k (\bbq),\mu_k (\tB) )$. That this yields another compatible pair is a
straightforward verification.
\begin{proposition}[\protect{\cite[Proposition 2.6]{GoodearlYakimov2017QCA}}]\label{prop:mutation_preserves_good_things}

	Let $(\bbq, \tB)$ be a compatible pair, and $k \in \ex$. Assume that the principal part
	of $\tB$ is skew-symmetrizable. Then the pair $(\mu_k (\bbq), \mu_k(\tB))$ is
	compatible, the principal part of $\mu_k(\tB)$ is skew-symmetrizable, and the
	$\wt{\bbt}$-matrix of the new pair as in \cref{def:compatible_pair} is the same as the
	one for the pair $(\bbq, \tB)$.
\end{proposition}

\medskip

Although \cref{eq:quantum_exchange_naive} has led us on the right path, it is not
perfect. It depends heavily on the order in which we take the products. It would be
nice if there was some ``order-independent'' way to define the exchange relation. To do
this, we will introduce the \emph{based quantum torus}\index{torus!based quantum} and
\emph{toric frames}\index{toric frame}.

The idea is that we want to symmetrize or balance the equation $Y_i Y_j = q_{ij} Y_j
	Y_i$. To do this, we assume the existence of another multiplicatively skew-symmetric
matrix $\bbr \in M_N(\bbK^\times)$ such that $\bbq = \bbr^{\cdot 2} = (r_{ij}^2)$. We
then find that
\begin{equation*}
	\Omega_\bbr(e_i, e_j)\inv Y_i Y_j = r_{ji} Y_i Y_j = r_{ij} Y_j Y_i = \Omega_\bbr (e_j, e_i)\inv Y_j Y_i.
\end{equation*}
%
We then define
\begin{equation*}
	Y^{(e_i + e_j)} = \Omega_\bbr(e_i, e_j)\inv Y_i Y_j = \Omega_\bbr(e_j, e_i)\inv Y_i Y_j.
\end{equation*}
%
This new element $Y^{(e_i + e_j)}$ represents the order-independent product of $Y_i$
and $Y_j$. This leads to the \emph{based quantum torus}\index{torus!based quantum}
$\mcT_{\bbr^{\cdot 2}}$ which has a $\bbK$-basis consisting of elements
\begin{equation*}
	Y^{(f)} \coloneq \mcS_{\mathbf{r}}(f)Y^f = \mcS_{\mathbf{r}}(f)Y_1^{f_1}\dots Y_N^{f_N}
	\text{ for } f = (f_1, \dots, f_n)^T \in \bbZ^N,
\end{equation*}
where
\begin{equation*}
	\mcS_{\mathbf{r}}(f) \coloneq \prod_{i < j}r_{ij}^{-f_if_j}.
\end{equation*}
Then $Y^{(e_k)} = Y_k$ for all $k \in [1, N]$ as
\begin{align*}
	Y^{(e_k)}
	= \mcS_{\mathbf{r}}(e_k)Y^{e_k}
	= (\prod_{i < j}r_{ij}^{-\delta_{ik}\delta_{jk}} )Y_k
	= (\prod_{i < j}r_{ij}^0 )Y_k
	= Y_k.
\end{align*}
Furthermore, as desired, we have the following multiplication rule for $f,g \in \bbZ^N$:
\begin{equation*}
	\Omega_\bbr(f,g)\inv Y^{(f)}Y^{(g)} = Y^{(f+g)} = \Omega_\bbr (g, f)\inv Y^{(g)}Y^{(f)}.
\end{equation*}
Indeed,
\begin{align*}
	Y^{(f)}Y^{(g)}
	 & = \mcS_{\mathbf{r}}(f)Y^f\mcS_{\mathbf{r}}(g)Y^g                                    \\
	 & = (\prod_{i < j}r_{ij}^{-f_if_j})(\prod_{i}Y_i^{f_i})
	(\prod_{i < j}r_{ij}^{-g_ig_j})( \prod_{i}Y_i^{g_i})                                   \\
	 & = (\prod_{i < j}r_{ij}^{-f_if_j - g_ig_j})
	(\prod_{k}Y_k^{f_k})(\prod_{k}Y_k^{g_k})                                               \\
	 & = (\prod_{i < j}r_{ij}^{-f_if_j - g_ig_j})
	(\prod_{i < j}r_{ij}^{-2g_if_j})(\prod_{k}Y_k^{f_k + g_k})
	 & (Y_i^{f_i} Y_j^{g_j} = r_{ij}^{2f_ig_j}Y_j^{g_j}Y_i^{f_i})                          \\
	 & = (\prod_{i < j}r_{ij}^{-(f_if_j +2g_if_j + g_ig_j)})
	(\prod_{k}Y_k^{f_k + g_k})                                                             \\
	 & = (\prod_{i < j}r_{ij}^{f_ig_j - g_if_j})
	(\prod_{i < j}r_{ij}^{-(f_if_j + f_ig_j + g_if_j + f_ig_j)})(\prod_{k}Y_k^{f_k + g_k}) \\
	 & = (\prod_{i,j}r_{ij}^{f_ig_j})
	(\prod_{i < j}r_{ij}^{-(f_i + g_j)(f_j + g_j)})(\prod_{k}Y_k^{f_k + g_k})              \\
	 & = \Omega_{\mathbf{r}}(f,g)Y^{(f + g)}.
\end{align*}

\medskip

We now introduce one final object, which serves multiple purposes. It combines the data
of the matrix $\bbr$, the cluster variables $Y_1, \dots, Y_N$ and the based quantum
torus $\mcT_{\bbr^{\cdot 2}}$ into one object. Additionally, it linearizes all the
multiplicative identities. Another useful property is that it allows permuting the
cluster variables, which we will need later on.
\begin{definition}
	Let $\mcF$ be a division algebra over $\bbK$. A map $M \colon \bbZ^N \to \mcF$ is called a \emph{toric frame}\index{toric frame} if there exists
	a multiplicatively skew-symmetric matrix $\mathbf{r} \in M_N(\bbK^\times)$ such that:
	\begin{enumerate}
		\item There is an algebra embedding $\varphi \colon \mcT_{\mathbf{r}^{\cdot 2}} \injto \mcF$
		      given by $\varphi(Y_i) = M(e_i)$, such that $\mcF =
			      \Fract(\varphi(\mcT_{\mathbf{r}^{\cdot 2}}))$.
		\item For all $f \in \bbZ^N$, $M(f) = \varphi(Y^{(f)})$.
	\end{enumerate}
\end{definition}
\begin{remark}
	We can always recover the matrix $\mathbf{r}$ from the toric frame $M$,
	since
	\begin{align*}
		r_{ij}
		                         & = \Omega_{\mathbf{r}}(e_i, e_j)                    \\
		                         & = Y^{(e_i)}Y^{(e_j)}(Y^{(e_i + e_j)})\inv          \\
		                         & \Big\Updownarrow                                   \\
		r_{ij} = \varphi(r_{ij}) & = \varphi(Y^{(e_i)}Y^{(e_j)}(Y^{(e_i + e_j)})\inv) \\
		                         & = M(e_i)M(e_j)M(e_i + e_j)\inv.
	\end{align*}
	We'll use the notation $\mathbf{r}(M)$ to denote the matrix of the toric frame $M$.
\end{remark}
%
With the necessary background out of the way, it is time to define the quantum analogs
of seeds, mutations, and cluster algebras.
\begin{definition}
	A \emph{quantum seed}\index{seed!quantum} is a pair $(M, \tB)$ consisting of a toric frame $M$
	and an $N \times \ex$ matrix $\tB$ over $\bbZ$ such that
	\begin{enumerate}
		\item The principal part of $\tB$ is skew-symmetrizable.
		\item The pair $(\mathbf{r}(M), \tB)$ is compatible.
	\end{enumerate}
\end{definition}
\begin{remark}
	Note that the compatibility is asked between $\bbr$ and $\tB$, and not between $\bbq = \bbr^{\cdot 2}$ and $\tB$. This still implies compatibility of $(\bbq, \tB)$, because
	\begin{equation*}
		\Omega_\bbq(b^k, e_j) = \Omega_\bbr(b^k, e_j)^2 = \begin{dcases*}
			1,                          & if $j \neq k$ \\
			\text{not a root of unity}, & if $j = k$.
		\end{dcases*}
	\end{equation*}
\end{remark}

Mutation of quantum seeds is defined by the analogue of
\cref{eq:quantum_exchange_naive} for toric frames together with the mutation of
compatible pairs. That this works is ensured by the following proposition.
\begin{proposition}[\protect{\cite[Proposition 2.9, Corollary 2.11]{GoodearlYakimov2017QCA}}]\label{prop:formula_for_mutation}

	For all quantum seeds $(M, \tB)$ of $\mcF$ and $k\in \ex$, there exists a toric frame
	$\mu_k(M)$ whose matrix equals $\mu_k(\bbr(M))$ (cf. \cref{eq:def_mutation_q}). This
	toric frame satisfies
	\begin{align*}
		\mu_k(M)(e_j) & = M(e_j), \text{ for all } j \neq k,         \\
		\mu_k(M)(e_k) & = M(-e_k + [b^k]_{+}) + M(-e_k - [b^k]_{-}).
	\end{align*}
	%
	The \emph{mutation of the quantum seed}\index{mutation!quantum seed} $(M, \tB)$ is then
	defined as $(\mu_k(M), \mu_k(\tB))$. This is again a quantum seed, and mutation defined
	this way is involutive:
	\begin{equation*}
		\mu^2_k (M, \tB) = (M, \tB).
	\end{equation*}
\end{proposition}
\begin{remark}

	Because $\mu_k(\bbr)^{\cdot 2} = \mu_k(\bbq)$ where $\mu_k(\bbq)$ is defined via
	\cref{eq:def_mutation_q}, this definition does what we expected for the quantum torus
	$\mcT_\bbq = \mcT_{\bbr^{\cdot 2}}$.
\end{remark}

\section{Quantum cluster algebras}

Let $\binv \subseteq [1, N]\setminus \ex$ be some subset. This set will determine which
of the frozen variables will also be inverted in the cluster algebra. We then define
the \emph{quantum cluster algebra}\index{algebra!quantum cluster} $\mcA(M, \tB,
	\binv)_\bbK$ completely analogously to the classical setting. Namely, it is the
$\bbK$-subalgebra of $\mcF$ generated by all the cluster variables $M'(e_j), \, j \in
	[1, N]$ and the inverses $M'(e_l)\inv, \, l \in \binv$ for all the quantum seeds $(M',
	\tB')$ mutation-equivalent to $(M, \tB)$.

Write
\begin{equation*}
	\mcT\mcA_{(M, \tB)} = \bbK\langle M(e_l)^{\pm 1}, M(e_j) \mid l \in \ex \sqcup \binv,\, j \in [1, N] \setminus (\ex \sqcup \binv)\rangle.
\end{equation*}
%
We note two special cases. If $\ex \sqcup \binv = \emptyset$, then $\mcT\mcA_{(M, \tB)}
	= \mcA_{\bbr(M)^{\cdot 2}}$, while if $\binv = [1, N] \setminus \ex$, then
$\mcT\mcA_{(M, \tB)} = \mcT_{\bbr(M)^{\cdot 2}}$. This explains the notation. The
\emph{upper quantum cluster algebra}\index{algebra!upper quantum cluster} $\mcU(M, \tB,
	\binv)_\bbK$ is then defined as the intersection
\begin{equation*}
	\mcU(M, \tB, \binv) = \bigcap_{(M', \tB')} \mcT\mcA_{(M', \tB')} ,
\end{equation*}
%
where the intersection runs over all quantum seeds $(M', \tB')$ mutation-equivalent to
the initial seed $(M, \tB)$. We now have the following generalization of the Laurent
phenomenon (\cref{thm:laurent_phenomenon}), which we will refer to as the quantum
Laurent phenomenon.
\begin{theorem}[\protect{\cite[Theorem 2.15]{GoodearlYakimov2017QCA}}]\label{thm:quantum_laurent}

	For all quantum seeds $(M, \tB)$ of a division algebra $\mcF$ over $\bbK$ and subsets
	$\binv \subseteq [1, N] \setminus \ex$, we have the inclusion
	\begin{equation*}
		\mcA(M, \tB, \binv)_{\bbK} \subseteq \mcU(M, \tB, \binv)_\bbK.
	\end{equation*}
\end{theorem}

We finish this section by looking at rank 2 quantum cluster algebras. Fix a field
$\bbK$. There is really only one possible choice for the matrices $\bbr, \bbq \in
	M_2(\bbK^\times)$. Namely,
\begin{equation*}
	\bbr = \begin{pmatrix}
		1 & q\inv \\
		q & 1
	\end{pmatrix}, \quad
	\bbq = \begin{pmatrix}
		1   & q^{-2} \\
		q^2 & 1
	\end{pmatrix},
\end{equation*}
%
for some $q \in \bbK^\times$. In the based quantum torus, $\mcT_{\bbr^{\cdot 2}}$, we
have the basis consisting of elements
\begin{equation*}
	Y^{(f)} = \mcS_{\bbr}(f)Y^f = q^{f_1f_2}Y_1^{f_1}Y_2^{f_2}, \text{ for all } f = (f_1, f_2)^T \in \bbZ^2.
\end{equation*}
%
The elements $Y_1$ and $Y_2$ satisfy
\begin{equation*}
	Y_2 Y_1 = q^2 Y_1 Y_2.
\end{equation*}
To construct a quantum seed we take $m,n \in \bbZ_{> 0}$ and
\begin{equation*}
	\tB = B = \begin{pmatrix}
		0 & -m \\
		n & 0
	\end{pmatrix}.
\end{equation*}
%
It is skew-symmetrizable by $n$ and $m$. The pair $(\bbr, \tB)$ is compatible, as long
as $q$ is not a root of unity. Indeed,
\begin{equation*}
	\wt{\bbt} = \begin{pmatrix}
		q^n & 1   \\
		1   & q^m
	\end{pmatrix}.
\end{equation*}
%
Let $\mcF = \Fract(\mcT_\bbq)$, and define the toric frame $M \colon \bbZ^2 \to \mcF$
by $M(f) = Y^{(f)}$. Then the pair $(M, \tB)$ defines a quantum seed.

We will now look at mutations of this quantum seed. Mutating in direction 1 gives
\begin{align*}
	\mu_1(M)(e_1) = M_1(e_1)
	 & = M(-e_1 + n e_2) + M(-e_1)        \\
	 & = q^{-n}Y_1^{-1}Y_2^n + Y_1\inv    \\
	 & = Y_1\inv (1 + q^{-n}Y_2^n) = Y_3,
\end{align*}
and
\begin{equation*}
	\mu_1(\tB) = \tB_1 = \begin{pmatrix}
		0  & m \\
		-n & 0
	\end{pmatrix}.
\end{equation*}
%
Of course $M_1(e_2) = M(e_2) = Y_2$. The other mutation gives $\mu_2(M)(e_1) = M(e_1) =
	Y_1$,
\begin{align*}
	\mu_2(M)(e_2) & = M(-e_2) + M(-e_2 - m e_1)     \\
	              & = Y_2\inv + q^m Y_1^{-m}Y_2\inv \\
	              & = (1 + q^mY_1^{-m})Y_2\inv,
\end{align*}
and
\begin{equation*}
	\mu_2(\tB) = \tB_2 = \begin{pmatrix}
		0  & m \\
		-n & 0
	\end{pmatrix}.
\end{equation*}
%
We now consider the chain of mutations $\mu_2 \circ \mu_1$. On the one hand,
\begin{equation*}
	M_{12}(e_1) = M_1(e_2) = Y_3 = Y_1 \inv (1 + q^{-n}Y_2^n),
\end{equation*}
and on the other hand
\begin{align*}
	M_{12}(e_2)
	 & = M_1(-e_2 + m e_1) + M_1(-e_2)                          \\
	 & = q^{m}Y_3^m Y_2\inv + Y_2\inv                           \\
	 & = (1 + q^m Y_3^m) Y_2\inv                                \\
	 & = (1 + q^m (Y_1 \inv (1 + q^{-n}Y_2^n))^m)Y_2\inv = Y_4.
\end{align*}
%
For the matrix $\tB$, we have $\mu_2 (\mu_1(\tB)) = \tB_{12} = \tB$. Note that the
expressions quickly got messy, even in this simple example. To simplify matters, we
will from now on work with $n = m = 1$. Then
\begin{equation*}
	Y_3 = Y_1\inv (1 + q\inv Y_2),
\end{equation*}
and
\begin{align*}
	Y_4
	 & = (1 + Y_1 \inv q (1 + q\inv Y_2))Y_2\inv \\
	 & = (1 + q Y_1\inv + Y_1\inv Y_2)Y_2\inv    \\
	 & = Y_1\inv (Y_1 + q + Y_2)Y_2\inv.
\end{align*}
%
We continue the chain of mutations alternating between 1 and 2.
\begin{align*}
	Y_5
	 & = Y_3\inv (1 + q\inv Y_4)                                                  \\
	 & = (1 + q\inv Y_2)\inv Y_1 (1 + q\inv Y_1\inv( Y_1 + q + Y_2)Y_2\inv)       \\
	 & = (1 + q\inv Y_2)\inv  (Y_1 + q\inv ( Y_1 + q + Y_2)Y_2\inv)               \\
	 & = (1 + q\inv Y_2)\inv  (Y_1 Y_2 + q\inv Y_1 + 1 + q\inv Y_2)Y_2\inv        \\
	 & = (1 + q\inv Y_2)\inv  (q^{-2} Y_2 Y_1 + q\inv Y_1 + 1 + q\inv Y_2)Y_2\inv \\
	 & = (1 + q\inv Y_2)\inv  (1 + q\inv Y_2)(q\inv Y_1 + 1)Y_2\inv               \\
	 & = (q\inv Y_1 + 1)Y_2\inv.
\end{align*}
%
Just like in the classical setting, we have a dramatic simplification! Continuing on,
we find
\begin{align*}
	Y_6
	 & = (1 + q Y_5)Y_4 \inv                                      \\
	 & = (1 + q(q\inv Y_1 + 1)Y_2\inv)Y_2 (Y_1 + q + Y_2)\inv Y_1 \\
	 & = (Y_2 +  Y_1 + q)Y_2\inv Y_2 (Y_1 + q + Y_2)\inv Y_1      \\
	 & = Y_1,
\end{align*}
%
and
\begin{align*}
	Y_7 & = Y_5\inv (1 + q\inv Y_6)                 \\
	    & = Y_2 (1 + q\inv Y_1)\inv (1 + q\inv Y_1) \\
	    & = Y_2.
\end{align*}
%

It seems that quantum cluster algebras behave similarly to the classical case. Indeed,
this is partly true. For a given quantum seed $(M, \tB)$, define the \emph{exchange
	graph}\index{graph!exchange} as the graph with mutation-equivalent quantum seeds as
vertices, and mutations as edges. Then the exchange graph for the quantum cluster
algebra will equal the exchange graph for the classical cluster algebra, obtained by
taking the ``limit'' $\bbq \to \mathbf{1}$ (\cite[Theorem
	6.1]{BerensteinZelevinsky2005QCA})\footnote{This statement is not very precisely
	formulated, but should at least give some intuition. Because we will not come back to
	this, we leave it as is.}. As a consequence, the quantum cluster algebra will be
cluster-finite precisely when the classical cluster algebra corresponding to $\tB$ is
cluster-finite! This is the phenomenon that we observed in the example. Setting $q = 1$
in all the equations above, gives precisely the cluster variables coming from the
classical cluster algebra!

\section{Quantum Grassmannians}
???

