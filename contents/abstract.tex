\begin{abstract}

	Cluster algebras were introduced by \textcite{FominZelevinsky2002CAF} as a
	combinatorial and algebraic framework to study dual canonical bases and total
	positivity in semisimple groups. It was quickly realized that these structures are
	useful in many other contexts, and the field has rapidly grown beyond this initial
	purpose. In this thesis, we provide an introduction to the theory of classical cluster
	algebras, and explore different settings in which they appear. Additionally, we look at
	a noncommutative generalization of cluster algebras, so-called quantum cluster
	algebras. In particular, following the work of \textcite{GoodearlYakimov2017QCA}, we
	look at CGL extensions, a type of equivariant iterated Ore extensions, and show that
	these always carry the structure of a quantum cluster algebra.
\end{abstract}

\begin{otherlanguage}{dutch}
	\begin{abstract}

		Clusteralgebra's werden geïntroduceerd door \textcite{FominZelevinsky2002CAF} als een
		combinatorisch en algebraïsch kader om canonieke basissen en totale positiviteit in
		semisimpele groepen te bestuderen. Het werd snel duidelijk dat deze structuren ook
		nuttig zijn in andere contexten, en het domein is snel gegroeid buiten het
		oorspronkelijk doel. In deze thesis, geven we een introductie tot de theorie van de
		klassieke clusteralgebra's, en onderzoeken we verschillende plaatsen waar ze voorkomen.
		Bovendien kijken we naar een niet commutatieve veralgemening van clusteralgebra's,
		zogenaamde kwantum clusteralgebra's. In het bijzonder, door het werk van
		\textcite{GoodearlYakimov2017QCA} te volgen, kijken we naar CGL-uitbreidingen, een vorm van
		equivariante geïtereerde Ore uitbreidingen, en tonen we aan dat deze altijd de structuur
		van een kwantum clusteralgebra hebben.
	\end{abstract}

\end{otherlanguage}
