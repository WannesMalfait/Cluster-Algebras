\chapter*{Introduction}
\addcontentsline{toc}{chapter}{Introduction}

As the title suggests, cluster algebras and the different contexts in which they appear
will be the central topic of this thesis. The original motivation for cluster algebras
was to use their structure to better understand canonical bases and the notion of total
positivity for semisimple groups. In this thesis we will not focus on this aspect of
cluster algebras. Instead, we will look at different methods of constructing cluster
algebras, and give an overview of the main results in the theory.

A \emph{cluster algebra} is a type of commutative algebra with a set of distinguished
generators called \emph{cluster variables}. These are grouped into \emph{clusters},
which are related to each other by a type of \emph{exchange relation}: If $x$ is a
cluster variable of some cluster $\bx$, then a new cluster is formed by replacing $x$
with $x'$ where
\begin{equation}\label{eq:exchange_general}
	x x' = M_{1} + M_{2}\tag{$*$},
\end{equation}
%
for some monomials $M_1$ and $M_2$ in the other cluster variables of $\bx$. Remarkably,
these structures are both flexible enough to be applicable to a large class of
algebras, and powerful enough to make strong statements about the algebra.

\medskip

In \cref{sec:classical_cluster_algebras}, we introduce cluster algebras from the ground
up. In essence, a cluster algebra is determined by the monomials that appear in
\cref{eq:exchange_general}. There are two aspects to this: the coefficients and the
exponents. We start in \cref{sec:cluster_algebras_quivers} by looking at the simplest
kinds of cluster algebras. There the coefficients of the monomials are trivial, while
the exponents are determined by a \emph{quiver}, i.e., a directed graph with no
1-cycles or 2-cycles. This setting allows us to get familiar with the general concepts
without being bogged down by other auxiliary definitions. The general setting is worked
out in \cref{sec:ice_quivers_and_coefficients}.

Finally, in \cref{sec:cluster_algebras_surfaces}, we look at a way of associating a
cluster algebra to a surface. For these cluster algebras, the clusters are in
one-to-one correspondence with triangulations of the surface. As a special instance, we
look at the triangulations of a polygon, and show that cluster algebra can be
identified with the homogeneous coordinate ring of a certain Grassmannian.

\medskip

We now leave the realm of commutative algebras, and look at quantum cluster algebras.
In \cref{sec:quantum_cluster_algebras}, we look at the modifications necessary to
obtain a noncommutative version of a cluster algebra. Remarkably, many of the
structural results from the classical setting carry over to the quantum setting. We
explore the similarities between the two in \cref{sec:quantum_grassmannian}, by
studying quantum Grassmannians and comparing the quantum cluster algebra structure on
them to that of the classical cluster algebra structure for the Grassmannian.

\medskip

The last part, \cref{sec:cgl_extensions}, focuses on one specific class of
noncommutative algebras, the so-called CGL extensions. These are a type of iterated Ore
extension
\begin{equation*}
	R = \bbK[x_1][x_2; \sigma_2, \delta_2]\cdots[x_N; \sigma_N, \delta_N],
\end{equation*}
%
equipped with a special type of action by a $\bbK$-torus $(\bbK^\times)^r$. It turns
out that, under some minor assumptions, these always carry the structure of a quantum
cluster algebra. Furthermore, the quantum cluster algebra structure can be described
explicitly in terms of ring-theoretic properties of CGL extensions.

In \cref{sec:cgl_introduction} we define CGL extensions and motivate why one should
expect a quantum cluster algebra structure. The next section,
\cref{sec:formulating_theorem} is focused on pinning down the exact conditions needed,
and formulating a precise statement of the quantum cluster algebra structure.
\Cref{sec:normalizing_generators,sec:constructing_quantum_seeds,sec:upper_cluster_algebra}
then contain the proof of this statement, split into three parts. The first,
\cref{sec:normalizing_generators} is the most technical, and essentially proves the
inductive step. The second, \cref{sec:constructing_quantum_seeds}, uses this to
inductively construct the clusters. Finally, in \cref{sec:upper_cluster_algebra} we
prove that the quantum cluster algebra equals the CGL extension.

\section*{Acknowledgements}

First and foremost I would like to thank my two supervisors, Kenny De Commer and
Geoffrey Janssens. Before the start of this academic year, I had never heard of a
cluster algebra. With your mini-lectures, seminars, and toy example problems, I was
quickly brought up to date. Furthermore, you provided a lot of additional context to
the theory, even if not required for the thesis itself.

Additionally, a big thank you to Andreas Lorrain for all the help with using the
\texttt{tikz} package, without which I would have spent countless more hours struggling
with cryptic error messages. Finally, a thank you to all the other master students as
well, for the moral support of working together in the same office on the thesis, and
figuring out all the administrative requirements.
