\documentclass{article}

%% PACKAGES

% Fancy chapters with toc
\usepackage{titlesec,titletoc}

\usepackage[style=alphabetic, maxnames=4]{biblatex}
\addbibresource{references.bib}

\usepackage{amsmath,amsfonts,amsthm,amssymb,mathtools}
\usepackage[shortlabels]{enumitem}
\usepackage{imakeidx}
\makeindex[intoc] % Add index to bibliography.
\usepackage[bookmarksdepth=2]{hyperref}
\usepackage{xcolor}
\hypersetup{
	colorlinks=true,
	linkcolor={red!50!black},
	citecolor={blue!50!black},
	urlcolor={blue!80!black}
}
\usepackage{cleveref}
\usepackage{tikz-cd}
\usepackage{aligned-overset}
\usepackage{tcolorbox}
\tcbuselibrary{breakable}
% \tcbuselibrary{theorems}
\tcbuselibrary{skins}
\usepackage{microtype}
\usepackage{nicematrix}
\NiceMatrixOptions{cell-space-limits = 5pt}

\renewcommand{\thechapter}{\Roman{chapter}}
\counterwithout{section}{chapter}
\counterwithout{figure}{chapter}
\counterwithout{table}{chapter}
\renewcommand*\thesection{\arabic{section}}

\titleformat{\chapter}[display]
{\bfseries\Large}
{\filleft
	\MakeUppercase{\chaptertitlename} \Huge\thechapter}
{3ex}
{\titlerule
	\vspace{2ex}%
	\filright}
[{%
			\vspace{2ex}%
			\titlerule
		}]

\titlecontents{chapter}
[0pt]
{\addvspace{1pc}}%
{\contentsmargin{0pt}%
	\bfseries
	\makebox[0pt][r]{\huge\thecontentslabel\enspace}%
	\large}
{% \addvspace{.2pc}%
	\contentsmargin{0pt}%
	\large}
{}
[\addvspace{.5pc}]

\newcommand{\chaptertoc}{%
	\dotfill
	\vspace*{1ex}
	\startcontents[chapters]
	\printcontents[chapters]{}{1}[2]{}
	\vspace*{1ex}
	\noindent\dotfill\\
	\vspace*{1pc}
}
% \usepackage{fancyhdr}
% \pagestyle{fancy}

\usetikzlibrary{positioning,quotes,calc,patterns}

%% Equation numbered by section
\numberwithin{equation}{section}

%% Theorem environments
\theoremstyle{plain}
\newtheorem{theorem}{Theorem}[section]
\newtheorem{lemma}[theorem]{Lemma}
\newtheorem{proposition}[theorem]{Proposition}
\newtheorem{corollary}[theorem]{Corollary}

\newtheorem{condition}{Condition}
\renewcommand*{\thecondition}{\Alph{condition}}
\crefname{condition}{condition}{conditions}

\newtheorem*{convention}{Convention}
\newtheorem*{theorem*}{Theorem}
\newtheorem*{lemma*}{Lemma}
\newtheorem*{prop*}{Proposition}
\newtheorem*{corollary*}{Corollary}

\theoremstyle{definition}
\newtheorem{definition}[theorem]{Definition}
% \newtheorem{example}[theorem]{Example}
\newtheorem{exercise}[theorem]{Exercise}
\newtheorem*{notation}{Notation}
\newtheorem*{definition*}{Definition}
\newtheorem*{example*}{Example}
\newtheorem*{exercise*}{Exercise}

\theoremstyle{remark}
\newtheorem{remark}[theorem]{Remark}
\newtheorem*{remark*}{Remark}

\tcolorboxenvironment{notation}{%
	blanker,breakable,left=5mm,
	before skip=10pt,after skip=10pt,
	borderline west={1mm}{0pt}{gray}}

\usepackage{thmtools}

\declaretheoremstyle[
  sibling=theorem,
	style=definition,
  spaceabove=1em plus 0.75em minus 0.25em,
  spacebelow=1em plus 0.75em minus 0.25em,
  qed={\itshape End of example.}
]{exmpstyle}

\declaretheorem[
  style=exmpstyle,
  title=Example,
  refname={example,examples},
  Refname={Example,Examples}
]{example}

%===========================================================%
%The code below customises theorem numbering
%===========================================================%

% \theoremstyle{plain}
% \newtheorem{innercustomgenericplain}{\customgenericname}
% \providecommand{\customgenericname}{}
% \newcommand{\newcustomtheoremplain}[2]{%
% 	\crefname{#2}{#2}{#2s}%
% 	\newenvironment{#1}[1]
% 	{%
% 		\renewcommand\customgenericname{#2}%
% 		\crefalias{innercustomgenericplain}{#2}%
% 		\renewcommand\theinnercustomgenericplain{##1}%
% 		\innercustomgenericplain
% 	}
% 	{\endinnercustomgenericplain}
% }

% \theoremstyle{definition}
% \newtheorem{innercustomgenericdefinition}{\customgenericname}
% \providecommand{\customgenericname}{}
% \newcommand{\newcustomtheoremdefinition}[2]{%
% 	\crefname{#2}{#2}{#2s}%
% 	\newenvironment{#1}[1]
% 	{%
% 		\renewcommand\customgenericname{#2}%
% 		\crefalias{innercustomgenericdefinition}{#2}%
% 		\renewcommand\theinnercustomgenericdefinition{##1}%
% 		\innercustomgenericdefinition
% 	}
% 	{\endinnercustomgenericdefinition}
% }

% \theoremstyle{remark}
% \newtheorem{innercustomgenericremark}{\customgenericname}
% \providecommand{\customgenericname}{}
% \newcommand{\newcustomtheoremremark}[2]{%
% 	\crefname{#2}{#2}{#2s}%
% 	\newenvironment{#1}[1]
% 	{%
% 		\renewcommand\customgenericname{#2}%
% 		\crefalias{innercustomgenericremark}{#2}%
% 		\renewcommand\theinnercustomgenericremark{##1}%
% 		\innercustomgenericremark
% 	}
% 	{\endinnercustomgenericremark}
% }

% \newcustomtheoremplain{customtheorem}{Theorem}
% \newcustomtheoremplain{customlemma}{Lemma}
% \newcustomtheoremplain{customprop}{Proposition}
% \newcustomtheoremplain{customcorollary}{Corollary}

% \newcustomtheoremdefinition{customdefinition}{Definition}
% \newcustomtheoremdefinition{customexample}{Example}
% \newcustomtheoremdefinition{customexercise}{Exercise}

% \newcustomtheoremremark{customremark}{Remark}

\input{definitions/macros}
\input{definitions/letterfonts}

\newcommand{\ex}{\mathbf{ex}}
\newcommand{\bx}{\mathbf{x}}
\newcommand{\tbx}{\tilde{\bx}}
\newcommand{\tB}{\tilde{B}}
\DeclareMathOperator{\Fract}{Fract}

%%%%%%%%%%%%%
%% Fill in!%%
%%%%%%%%%%%%%
\title{Cluster Algebras}
\subtitle{Collection of notes on (quantum) cluster algebras}
\faculty{Sciences and Bio-Engineering Sciences}
\author{Wannes Malfait}
\date{Academic year 2023-2024}
\promotors{}
\pretitle{} %% Dissertation for the grade of Bachelor's in mathematics

\begin{document}

\maketitle
\newpage
\tableofcontents
\newpage

\section{Some definitions}

We start with the non-quantized version of cluster algebras,
introduced in \cite{FominZelivinsky2002CAF}.
We won't take the most general definition,
since that doesn't carry over as nicely to the quantized version.

\subsection{Notation}

Some of the notation that we will use throughout this paper.
Let $[a,b] = \{a, a+1, \dots, b\}$ for integers $a,b \in \bbZ$,
where $[a,b] = \emptyset$ if $a > b$.
Fix an integer $N$, and a subset $\ex \subseteq [1, N]$ of size $n$.
The elements of $\ex$ will be called the \emph{exchangeable} indices.
Let $\mcF = \bbQ(Y_1, \dots, Y_N)$, be the field of rational functions over $\bbQ$.
Boldface letters will be used to denote vectors, matrices, or clusters.

\subsection{Classical cluster algebras}

Before giving the definition of a cluster algebra,
we need to define what seeds and seed mutations are.
\begin{definition}[\cite{BerensteinZelivinsky2005QCA}]
    A \emph{seed} is a pair $(\tilde{\mathbf{x}}, \tilde{B})$ such that
    \begin{enumerate}
        \item $\tilde{\mathbf{x}}$ is a \emph{free generating set} of $\mcF$.
              So, $\tilde{\mathbf{x}}$ is a set of $N$ elements $x_1, \dots, x_n \in \mcF$ that are algebraically independent, and generate $\mcF$.
        \item $\tilde{B}$ is an $N \times n$ matrix over $\bbZ$ with columns labeled by the exchangeable
              indices $\ex$. Let $B$ be the $n \times n$ submatrix
              of $\tilde{B}$ consisting of all the rows with index in $\ex$.
              We call $B$ the \emph{principal part} of $\tilde{B}$, and require that it is \emph{skew-symmetrizable}.
              This means that there exists a diagonal matrix $D$ with positive entries
              such that $D\inv B D = - B^T$ i.e., $BD$ is skew-symmetric.
    \end{enumerate}
\end{definition}

Looking only at the exchangeable indices, we get the \emph{cluster}
$\mathbf{x} = \{x_i \mid i \in \ex \} \subseteq \tilde{\mathbf{x}}$.
\begin{definition}
    Let $(\tilde{\mathbf{x}}, \tilde{B})$ be a seed.
    The \emph{seed mutation} in direction $k \in \ex$
    produces a new seed $\mu_k(\tilde{\mathbf{x}}, \tB) = (\tilde{\bx}', \tB')$, where:
    \begin{itemize}
        \item $\tbx' = (\tbx \setminus \{x_k\}) \cup \{x_k'\}$,
              with $x_k' \in \mcF$ determined by the \emph{exchange relation}:
              \begin{equation}
                  \label{eq:exchange_relation}
                  x_kx_k' = \prod_{\substack{i \in [1,N] \\ b_{ik} > 0}}x_i^{b_{ik}} + \prod_{\substack{i \in [1, N] \\ b_{ik} < 0}}x_i^{-b_{ik}}.
              \end{equation}
        \item $\tB'$ is given by the formula:
              \begin{equation}
                  \label{eq:matrix_mutation}
                  b'_{ij} =
                  \begin{cases}
                      -b_{ij}                                            & \text{ if } i=k \text{ or } j=k \\
                      b_{ij} + \frac{|b_{ik}|b_{kj} + b_{ik}|b_{kj}|}{2} & \text{ otherwise}
                  \end{cases}
                  .
              \end{equation}
    \end{itemize}
\end{definition}

\begin{remark}
    These formulas seem a bit random at first sight,
    but have a nice interpretation in the language of quivers.
    Assume that $B$ is skew-symmetric,
    and let $Q$ be the directed graph on $N$ vertices with $b_{ij}$ edges
    from $i$ to $j$ if $j \in \ex$. If $b_{ij}$ is negative,
    the edges have the opposite orientation.
    Now, define the mutation of $Q$ at vertex $k$ as follows:
    \begin{itemize}
        \item For every edge connected to $k$, flip its orientation.
        \item For every pair of edges $i\to k$ and $k\to j$, create an edge $i \to j$.
              Then ``cancel'' any pair of edges between $i$ and $j$ with opposite orientation.
    \end{itemize}
    The new matrix representing this graph, will be given precisely by \cref{eq:matrix_mutation}.
    Furthermore, the exchange relation (\ref{eq:exchange_relation}) now takes the following form:
    \begin{equation*}
        x_kx_k' = \prod_{i \to k}x_i + \prod_{k \to j}x_i.
    \end{equation*}
\end{remark}

\medskip

With these definitions out of the way we define the \emph{cluster algebra} associated to a seed
as the subalgebra of $\mcF$ generated by the union of clusters of all seeds
obtained through (iterative) mutations of the initial seed.

Instead of proving things now for the classical case,
we will jump directly to the quantized version to
avoid having to prove the same results twice.

\subsection{Quantum cluster algebras}

\textbf{TODO: add more notation!}
We will follow the notation as presented in \cite{GoodearlYakimov2017QCA}.
Let $\bbK$ be any field.
As a first step, we move from $\bbQ(Y_1, \dots, Y_N)$ to
the \emph{quantum torus}
\begin{equation*}
    \mcT_{\mathbf r} =
    \frac{\bbK\langle Y_1^{\pm 1}, \dots, Y_N^{\pm 1} \rangle}{\langle Y_iY_j = r_{ij}Y_jY_i \rangle},
\end{equation*}
where $\mathbf{r} \in M_N(\bbK^*)$ is a multiplicatively skew-symmetric matrix i.e.,
$r_{ij} = r_{ji}\inv, r_{ii} = 1$.
Associated to any such matrix, is a skew-symmetric bicharacter
\footnote{With a bicharacter we mean a map from a direct product,
    such that each of the component maps is a group character.}:
\begin{equation*}
    \Omega_{\mathbf{r}} \colon \bbZ^N \times \bbZ^N \to \bbK^* \colon
    \Omega_{\mathbf{r}}(e_i, e_j) = r_{ij}.
\end{equation*}

To prevent formulas from having $\frac{1}{2}$'s in the exponents,
we will work with the \emph{quantum torus}\footnote{
    The reason it's called a quantum torus is the following:
    it is a quantization of the coordinate algebra of the torus $(\bbK^{*})^N$.
}
$\mcT_{\mathbf{r}^{\cdot 2}}$,
where
\begin{equation*}
    \mathbf{r}^{\cdot 2} \coloneq (r_{ij}^2) \in M_N(\bbK^*).
\end{equation*}
This torus has a $\bbK$-basis consisting of elements
\begin{equation*}
    Y^{(f)} \coloneq \mcS_{\mathbf{r}}(f)Y^f = \mcS_{\mathbf{r}}(f)Y_1^{m_1}\dots Y_N^{m_N}
    \text{ for } f = (m_1, \dots, m_n)^T \in \bbZ^N,
\end{equation*}
where
\begin{equation*}
    \mcS_{\mathbf{r}}(f) \coloneq \prod_{i < j}r_{ij}^{-m_im_j}.
\end{equation*}
Then $Y^{(e_k)} = Y_k$ for all $k \in [1, N]$ as
\begin{align*}
    Y^{(e_k)}
     & = \mcS_{\mathbf{r}}(e_k)Y^{e_k}                      \\
     & = (\prod_{i < j}r_{ij}^{-\delta_{ik}\delta{jk}} )Y_k \\
     & = (\prod_{i < j}r_{ij}^0 )Y_k                        \\
     & = Y_K.
\end{align*}
Additionally, we have the following multiplication rule for $f,g \in \bbZ^N$:
\begin{equation*}
    Y^{(f)}Y^{(g)} = \Omega_{\mathbf{r}}(f,g)Y^{(f+g)}.
\end{equation*}
Indeed,
\begin{align*}
    Y^{(f)}Y^{(g)}
     & = \mcS_{\mathbf{r}}(f)Y^f\mcS_{\mathbf{r}}(g)Y^g                                    \\
     & = (\prod_{i < j}r_{ij}^{-f_if_j})(\prod_{i}Y_i^{f_i})
    (\prod_{i < j}r_{ij}^{-g_ig_j})( \prod_{i}Y_i^{g_i})                                   \\
     & = (\prod_{i < j}r_{ij}^{-f_if_j - g_ig_j})
    (\prod_{k}Y_k^{f_k})(\prod_{k}Y_k^{g_k})                                               \\
     & = (\prod_{i < j}r_{ij}^{-f_if_j - g_ig_j})
    (\prod_{i < j}r_{ij}^{-2g_if_j})(\prod_{k}Y_k^{f_k + g_k})
     & (Y_i^{f_i} Y_j^{g_j} = r_{ij}^{2f_ig_j}Y_j^{g_j}Y_i^{f_i})                          \\
     & = (\prod_{i < j}r_{ij}^{-(f_if_j +2g_if_j + g_ig_j)})
    (\prod_{k}Y_k^{f_k + g_k})                                                             \\
     & = (\prod_{i < j}r_{ij}^{f_ig_j - g_if_j})
    (\prod_{i < j}r_{ij}^{-(f_if_j + f_ig_j + g_if_j + f_ig_j)})(\prod_{k}Y_k^{f_k + g_k}) \\
     & = (\prod_{i,j}r_{ij}^{f_ig_j})
    (\prod_{i < j}r_{ij}^{-(f_i + g_j)(f_j + g_j)})(\prod_{k}Y_k^{f_k + g_k})              \\
     & = \Omega_{\mathbf{r}}(f,g)Y^{(f + g)}.
\end{align*}
The torus $\mcT_{\mathbf{r}^{\cdot 2}}$ with the basis $\{Y^(f) \mid f \in \bbZ^N\}$
is called the \emph{based quantum torus} associated to the matrix $\mathbf{r}$.

We now come to the analog of clusters in the quantum setting.
\begin{definition}
    A map $M \colon \bbZ^N \to \mcF$ is called a \emph{toric frame} if there exists
    a multiplicatively skew-symmetric matrix $\mathbf{r} \in M_N(\bbK^*)$ such that:
    \begin{enumerate}
        \item There is an algebra embedding
              $\varphi \colon \mcT_{\mathbf{r}^{\cdot 2}} \injto \mcF$
              given by $\varphi(Y_i) = M(e_i)$, such that
              $\mcF = \Fract(\varphi(\mcT_{\mathbf{r}^{\cdot 2}}))$.
        \item For all $f \in \bbZ^N$, $M(f) = \varphi(Y^{(f)})$.
    \end{enumerate}
\end{definition}
\begin{remark}
    We can always recover the matrix $\mathbf{r}$ from the toric frame $M$,
    since
    \begin{align*}
        r_{ij}
                                 & = \Omega_{\mathbf{r}}(e_i, e_j)                    \\
                                 & = Y^{(e_i)}Y^{(e_j)}(Y^{(e_i + e_j)})\inv          \\
                                 & \Big\Updownarrow                                   \\
        r_{ij} = \varphi(r_{ij}) & = \varphi(Y^{(e_i)}Y^{(e_j)}(Y^{(e_i + e_j)})\inv) \\
                                 & = M(e_i)M(e_j)M(e_i + e_j)\inv.
    \end{align*}
    We'll use the notation $\mathbf{r}(M)$ to denote the matrix of the toric frame $M$.
\end{remark}
Now, for quantum seeds, we'll need an extra compatibility condition:
\begin{definition}
    A \emph{quantum seed} is a pair $(M, \tB)$ consisting of a toric frame $M$
    and an $N \times n$ matrix $B$ over $\bbZ$ such that
    \begin{enumerate}
        \item The principal part of $\tB$ is skew-symmetrizable.
        \item The pair $(\mathbf{r}(M), \tB)$ is \emph{compatible}.
    \end{enumerate}
\end{definition}

So, what is this extra compatibility?
In analogy with \cref{eq:exchange_relation} we want to define $\mu_k(M)$ in such
a way that $\mu_k(M)(e_j) = M(e_j)$ for $j \neq k$ and
\begin{equation*}
    M(e_k)\mu_k(M)(e_k) = M([b^k]_+) + M(-[b^k]_-),
\end{equation*}
where $b^k$ is the $k$-th column of $B$.
Bringing $M(e_k)$ to the other side,
and ignoring eventual symmetrization constants we can then define
\begin{equation*}
    \mu_k(M)(e_k) = M(-e_k + [b^k]_+) + M(-e_k - [b^k]_-).
\end{equation*}
Let $\epsilon$ be a sign $\pm 1$. Then we can rewrite the definition as
\begin{align*}
    \mu_k(M)(e_k)
     & = M(-e_k + [b^k]_+) + M(-e_k - [b^k]_-)                                    \\
     & = M(-e_k + [-\epsilon b^k]_+) + M(-e_k - [-\epsilon b^k]_+ + \epsilon b^k) \\
     & = M(E_\epsilon e_k) + M(E_\epsilon(e_k + \epsilon b^k)),
\end{align*}
where $E_\epsilon \in \GL_N(\bbZ)$ is the integer matrix with columns:
\begin{equation*}
    E_\epsilon = (e_1, \dots ,e_{k-1}, [-\epsilon b^k]_+ -e_k, e_{k+1}, \dots, e_N).
\end{equation*}
Since, $E_\epsilon$ maps the basis $e_1, \dots, e_N$ of $\bbZ^N$ to another basis,
and $E_\epsilon e_j = e_j$ for all $j \neq k$,
we can ignore it when checking that $\mu_k(M)$ defines a toric frame.
In other words,
we just need that $M' : \bbZ^N \to \mcF$ defines a toric frame, where
\begin{align*}
    M'(e_j) = M(e_j), j\neq k \quad M'(e_k) = M(e_k) + M(e_k + \epsilon b^k).
\end{align*}
On the one hand we have
\begin{align*}
    M'(e_k)M'(e_j)
     & = (M(e_k) + M(e_k + \epsilon b^k))M(e_j)                                                                                \\
     & = M(e_j)(\Omega_{\mathbf{r}}(e_k, e_j)^2 M(e_k) + \Omega_{\mathbf{r}}(e_k + \epsilon b^k, e_j)^2 M(e_k + \epsilon b^k)) \\
     & =\Omega_{\mathbf{r}}(e_k, e_j)^2  M(e_j)(M(e_k) + \Omega_{\mathbf{r}}(\epsilon b^k, e_j)^2 M(e_k + \epsilon b^k))       \\
     & =\mathbf{r}_{k,j}^2  M(e_j)(M(e_k) + \Omega_{\mathbf{r}}(b^k, e_j)^{2 \epsilon} M(e_k + \epsilon b^k)),
\end{align*}
while $M'$ being a toric frame would imply
\begin{align*}
    M'(e_k)M'(e_j) = (\mathbf{r}'_{k,j})^2 M'(e_j)M'(e_k).
\end{align*}
The easiest way to make the two equations compatible,
is to demand $\mathbf{r}' = \mathbf{r}$,
and $\Omega_{\mathbf{r}}(b^k, e_j) = 1$ for $j \neq k$.
\textbf{TODO: is this the only possible choice?}
This leads to the following definition:
\begin{definition}
    Let $\tilde{\mathbf{t}} \in M_{n \times N}(\bbK^*)$ be the matrix with entries
    \begin{equation*}
        t_{ij} = \Omega_{\mathbf{r}}(b^i, e_j) = \prod_{l = 1}^N r_{lj}^{b_{li}},
    \end{equation*}
    for $j \in \mathbf{ex}$ and $j \in [1, N]$.
    We say that $B$ and $\mathbf{r}$ are \emph{compatible} if
    $t_{ij} = 1$ whenever $i \neq j$,
    and all the $t_{ii}$ are \emph{not} roots of unity.
\end{definition}
\textbf{TODO: How to explain the root of unity part? Is there a clean way?}

\bibliographystyle{plain}
\bibliography{references.bib}
\end{document}
