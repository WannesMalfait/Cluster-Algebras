\documentclass{article}

%% PACKAGES

% Fancy chapters with toc
\usepackage{titlesec,titletoc}

\usepackage[style=alphabetic, maxnames=4]{biblatex}
\addbibresource{references.bib}

\usepackage{amsmath,amsfonts,amsthm,amssymb,mathtools}
\usepackage[shortlabels]{enumitem}
\usepackage{imakeidx}
\makeindex[intoc] % Add index to bibliography.
\usepackage[bookmarksdepth=2]{hyperref}
\usepackage{xcolor}
\hypersetup{
	colorlinks=true,
	linkcolor={red!50!black},
	citecolor={blue!50!black},
	urlcolor={blue!80!black}
}
\usepackage{cleveref}
\usepackage{tikz-cd}
\usepackage{aligned-overset}
\usepackage{tcolorbox}
\tcbuselibrary{breakable}
% \tcbuselibrary{theorems}
\tcbuselibrary{skins}
\usepackage{microtype}
\usepackage{nicematrix}
\NiceMatrixOptions{cell-space-limits = 5pt}

\renewcommand{\thechapter}{\Roman{chapter}}
\counterwithout{section}{chapter}
\counterwithout{figure}{chapter}
\counterwithout{table}{chapter}
\renewcommand*\thesection{\arabic{section}}

\titleformat{\chapter}[display]
{\bfseries\Large}
{\filleft
	\MakeUppercase{\chaptertitlename} \Huge\thechapter}
{3ex}
{\titlerule
	\vspace{2ex}%
	\filright}
[{%
			\vspace{2ex}%
			\titlerule
		}]

\titlecontents{chapter}
[0pt]
{\addvspace{1pc}}%
{\contentsmargin{0pt}%
	\bfseries
	\makebox[0pt][r]{\huge\thecontentslabel\enspace}%
	\large}
{% \addvspace{.2pc}%
	\contentsmargin{0pt}%
	\large}
{}
[\addvspace{.5pc}]

\newcommand{\chaptertoc}{%
	\dotfill
	\vspace*{1ex}
	\startcontents[chapters]
	\printcontents[chapters]{}{1}[2]{}
	\vspace*{1ex}
	\noindent\dotfill\\
	\vspace*{1pc}
}
% \usepackage{fancyhdr}
% \pagestyle{fancy}

\usetikzlibrary{positioning,quotes,calc,patterns}

%% Equation numbered by section
\numberwithin{equation}{section}

%% Theorem environments
\theoremstyle{plain}
\newtheorem{theorem}{Theorem}[section]
\newtheorem{lemma}[theorem]{Lemma}
\newtheorem{proposition}[theorem]{Proposition}
\newtheorem{corollary}[theorem]{Corollary}

\newtheorem{condition}{Condition}
\renewcommand*{\thecondition}{\Alph{condition}}
\crefname{condition}{condition}{conditions}

\newtheorem*{convention}{Convention}
\newtheorem*{theorem*}{Theorem}
\newtheorem*{lemma*}{Lemma}
\newtheorem*{prop*}{Proposition}
\newtheorem*{corollary*}{Corollary}

\theoremstyle{definition}
\newtheorem{definition}[theorem]{Definition}
% \newtheorem{example}[theorem]{Example}
\newtheorem{exercise}[theorem]{Exercise}
\newtheorem*{notation}{Notation}
\newtheorem*{definition*}{Definition}
\newtheorem*{example*}{Example}
\newtheorem*{exercise*}{Exercise}

\theoremstyle{remark}
\newtheorem{remark}[theorem]{Remark}
\newtheorem*{remark*}{Remark}

\tcolorboxenvironment{notation}{%
	blanker,breakable,left=5mm,
	before skip=10pt,after skip=10pt,
	borderline west={1mm}{0pt}{gray}}

\usepackage{thmtools}

\declaretheoremstyle[
  sibling=theorem,
	style=definition,
  spaceabove=1em plus 0.75em minus 0.25em,
  spacebelow=1em plus 0.75em minus 0.25em,
  qed={\itshape End of example.}
]{exmpstyle}

\declaretheorem[
  style=exmpstyle,
  title=Example,
  refname={example,examples},
  Refname={Example,Examples}
]{example}

%===========================================================%
%The code below customises theorem numbering
%===========================================================%

% \theoremstyle{plain}
% \newtheorem{innercustomgenericplain}{\customgenericname}
% \providecommand{\customgenericname}{}
% \newcommand{\newcustomtheoremplain}[2]{%
% 	\crefname{#2}{#2}{#2s}%
% 	\newenvironment{#1}[1]
% 	{%
% 		\renewcommand\customgenericname{#2}%
% 		\crefalias{innercustomgenericplain}{#2}%
% 		\renewcommand\theinnercustomgenericplain{##1}%
% 		\innercustomgenericplain
% 	}
% 	{\endinnercustomgenericplain}
% }

% \theoremstyle{definition}
% \newtheorem{innercustomgenericdefinition}{\customgenericname}
% \providecommand{\customgenericname}{}
% \newcommand{\newcustomtheoremdefinition}[2]{%
% 	\crefname{#2}{#2}{#2s}%
% 	\newenvironment{#1}[1]
% 	{%
% 		\renewcommand\customgenericname{#2}%
% 		\crefalias{innercustomgenericdefinition}{#2}%
% 		\renewcommand\theinnercustomgenericdefinition{##1}%
% 		\innercustomgenericdefinition
% 	}
% 	{\endinnercustomgenericdefinition}
% }

% \theoremstyle{remark}
% \newtheorem{innercustomgenericremark}{\customgenericname}
% \providecommand{\customgenericname}{}
% \newcommand{\newcustomtheoremremark}[2]{%
% 	\crefname{#2}{#2}{#2s}%
% 	\newenvironment{#1}[1]
% 	{%
% 		\renewcommand\customgenericname{#2}%
% 		\crefalias{innercustomgenericremark}{#2}%
% 		\renewcommand\theinnercustomgenericremark{##1}%
% 		\innercustomgenericremark
% 	}
% 	{\endinnercustomgenericremark}
% }

% \newcustomtheoremplain{customtheorem}{Theorem}
% \newcustomtheoremplain{customlemma}{Lemma}
% \newcustomtheoremplain{customprop}{Proposition}
% \newcustomtheoremplain{customcorollary}{Corollary}

% \newcustomtheoremdefinition{customdefinition}{Definition}
% \newcustomtheoremdefinition{customexample}{Example}
% \newcustomtheoremdefinition{customexercise}{Exercise}

% \newcustomtheoremremark{customremark}{Remark}

\input{definitions/macros}
\input{definitions/letterfonts}

\newcommand{\ex}{\mathbf{ex}}
\newcommand{\bx}{\mathbf{x}}
\newcommand{\tbx}{\tilde{\bx}}
\newcommand{\tB}{\tilde{B}}

%%%%%%%%%%%%%
%% Fill in!%%
%%%%%%%%%%%%%
\title{Cluster Algebras}
\subtitle{Collection of notes on (quantum) cluster algebras}
\faculty{Sciences and Bio-Engineering Sciences}
\author{Wannes Malfait}
\date{Academic year 2023-2024}
\promotors{}
\pretitle{} %% Dissertation for the grade of Bachelor's in mathematics

\begin{document}

\maketitle
\newpage
\tableofcontents
\newpage

\section{Some definitions}

We start with the non-quantized version of cluster algebras,
introduced in \cite{FominZelivinsky2002CAF}.
We won't take the most general definition,
since that doesn't carry over as nicely to the quantized version.

\subsection{Notation}

Some of the notation that we will use throughout this paper.
Let $[a,b] = \{a, a+1, \dots, b\}$ for integers $a,b \in \bbZ$,
where $[a,b] = \emptyset$ if $a > b$.
Fix an integer $N$, and a subset $\ex \subseteq [1, N]$ of size $n$.
The elements of $\ex$ will be called the \emph{exchangeable} indices.
Let $\mcF = \bbQ(Y_1, \dots, Y_N)$, be the field of rational functions over $\bbQ$.
Boldface letters will be used to denote vectors, matrices, or clusters.

\subsection{Classical cluster algebras}

Before giving the definition of a cluster algebra,
we need to define what seeds and seed mutations are.
\begin{definition}[\cite{BerensteinZelivinsky2005QCA}]
    A \emph{seed} is a pair $(\tilde{\mathbf{x}}, \tilde{B})$ such that
    \begin{enumerate}
        \item $\tilde{\mathbf{x}}$ is a \emph{free generating set} of $\mcF$.
              So, $\tilde{\mathbf{x}}$ is a set of $N$ elements $x_1, \dots, x_n \in \mcF$ that are algebraically independent, and generate $\mcF$.
        \item $\tilde{B}$ is an $N \times n$ matrix over $\bbZ$ with columns labeled by the exchangeable
              indices $\ex$. Let $B$ be the $n \times n$ submatrix
              of $\tilde{B}$ consisting of all the rows with index in $\ex$.
              We call $B$ the \emph{principal part} of $\tilde{B}$, and require that it is \emph{skew-symmetrizable}.
              This means that there exists a diagonal matrix $D$ with positive entries
              such that $D\inv B D = - B^T$ i.e., $BD$ is skew-symmetric.
    \end{enumerate}
\end{definition}

Looking only at the exchangeable indices, we get the \emph{cluster}
$\mathbf{x} = \{x_i \mid i \in \ex \} \subseteq \tilde{\mathbf{x}}$.
\begin{definition}
    Let $(\tilde{\mathbf{x}}, \tilde{B})$ be a seed.
    The \emph{seed mutation} in direction $k \in \ex$
    produces a new seed $\mu_k(\tilde{\mathbf{x}}, \tB) = (\tilde{\bx}', \tB')$, where:
    \begin{itemize}
        \item $\tbx' = (\tbx \setminus \{x_k\}) \cup \{x_k'\}$,
              with $x_k' \in \mcF$ determined by the \emph{exchange relation}:
              \begin{equation}
                  \label{eq:exchange_relation}
                  x_kx_k' = \prod_{\substack{i \in [1,N] \\ b_{ik} > 0}}x_i^{b_{ik}} + \prod_{\substack{i \in [1, N] \\ b_{ik} < 0}}x_i^{-b_{ik}}.
              \end{equation}
        \item $\tB'$ is given by the formula:
              \begin{equation}
                  \label{eq:matrix_mutation}
                  b'_{ij} =
                  \begin{cases}
                      -b_{ij}                                            & \text{ if } i=k \text{ or } j=k \\
                      b_{ij} + \frac{|b_{ik}|b_{kj} + b_{ik}|b_{kj}|}{2} & \text{ otherwise}
                  \end{cases}
                  .
              \end{equation}
    \end{itemize}
\end{definition}

\begin{remark}
    These formulas seem a bit random at first sight,
    but have a nice interpretation in the language of quivers.
    Assume that $B$ is skew-symmetric,
    and let $Q$ be the directed graph on $N$ vertices with $b_{ij}$ edges
    from $i$ to $j$ if $j \in \ex$. If $b_{ij}$ is negative,
    the edges have the opposite orientation.
    Now, define the mutation of $Q$ at vertex $k$ as follows:
    \begin{itemize}
        \item For every edge connected to $k$, flip its orientation.
        \item For every pair of edges $i\to k$ and $k\to j$, create an edge $i \to j$.
              Then ``cancel'' any pair of edges between $i$ and $j$ with opposite orientation.
    \end{itemize}
    The new matrix representing this graph, will be given precisely by \cref{eq:matrix_mutation}.
    Furthermore, the exchange relation (\ref{eq:exchange_relation}) now takes the following form:
    \begin{equation*}
        x_kx_k' = \prod_{i \to k}x_i + \prod_{k \to j}x_i.
    \end{equation*}
\end{remark}

\medskip

With these definitions out of the way we define the \emph{cluster algebra} associated to a seed
as the subalgebra of $\mcF$ generated by the union of clusters of all seeds
obtained through (iterative) mutations of the initial seed.

Instead of proving things now for the classical case,
we will jump directly to the quantized version to
avoid having to prove the same results twice.

\subsection{Quantum cluster algebras}

We will follow the notation as presented in \cite{GoodearlYakimov2017QCA}.

\bibliographystyle{plain}
\bibliography{references.bib}
\end{document}
