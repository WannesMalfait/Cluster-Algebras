\documentclass[aspectratio=169]{beamer}
\newcommand{\inv}{^{-1}}
\newcommand{\defeq}{\overset{\mathrm{def}}{=}}

\newcommand{\liff}{\leftrightarrow}
\newcommand{\lthen}{\rightarrow}
\newcommand{\opname}{\operatorname}
\newcommand{\surjto}{\twoheadrightarrow}
\newcommand{\injto}{\hookrightarrow}
\DeclareMathOperator{\img}{im} % Image
\DeclareMathOperator{\Img}{Im} % Image
\DeclareMathOperator{\coker}{coker} % Cokernel
\DeclareMathOperator{\Coker}{Coker} % Cokernel
\DeclareMathOperator{\Ker}{Ker} % Kernel
\DeclareMathOperator{\rank}{rank} % rank
\DeclareMathOperator{\Spec}{Spec} % spectrum
\DeclareMathOperator{\Tr}{Tr} % trace
\DeclareMathOperator{\pr}{pr} % projection
\DeclareMathOperator{\ext}{ext} % extension
\DeclareMathOperator{\pred}{pred} % predecessor
\DeclareMathOperator{\dom}{dom} % domain
\DeclareMathOperator{\cod}{cod} % codomain
\DeclareMathOperator{\ran}{ran} % range
\DeclareMathOperator{\Hom}{Hom} % homomorphism
\DeclareMathOperator{\Mor}{Mor} % morphisms
\DeclareMathOperator{\ob}{ob} % objects
\DeclareMathOperator{\mor}{mor} % morphisms
\DeclareMathOperator{\Fun}{Fun} % functors
\DeclareMathOperator{\Nat}{Nat} % natural transformations
\DeclareMathOperator{\End}{End} % endomorphism
\DeclareMathOperator{\Ann}{Ann} % annihilator
\DeclareMathOperator{\lt}{lt} % leading term
\DeclareMathOperator{\Fract}{Fract}
\DeclareMathOperator{\id}{id}
\DeclareMathOperator{\supp}{supp}

% Category Theory
\DeclareMathOperator{\Mod}{\mathbf{Mod}}
\DeclareMathOperator{\Top}{\mathbf{Top}}
\DeclareMathOperator{\Vect}{\mathbf{Vect}}
\DeclareMathOperator{\Ab}{\mathbf{Ab}}
\DeclareMathOperator{\Set}{\mathbf{Set}}
\DeclareMathOperator{\Sh}{\mathbf{Sh}}
\DeclareMathOperator{\PSh}{\mathbf{PSh}}

\newcommand{\ol}{\overline}
\newcommand{\ul}{\underline}
\newcommand{\wt}{\widetilde}
\newcommand{\wh}{\widehat}
\newcommand{\norm}[1]{\left\| #1 \right\|}
\newcommand{\inner}[2]{\left\langle #1 , #2 \right\rangle}


% Things Lie
\newcommand{\kb}{\mathfrak b}
\newcommand{\kg}{\mathfrak g}
\newcommand{\kh}{\mathfrak h}
\newcommand{\kn}{\mathfrak n}
\newcommand{\ku}{\mathfrak u}
\newcommand{\kz}{\mathfrak z}
\DeclareMathOperator{\Ext}{Ext} % Ext functor
\DeclareMathOperator{\Tor}{Tor} % Tor functor
\newcommand{\gl}{\opname{\mathfrak{gl}}} % frak gl group
\renewcommand{\sl}{\opname{\mathfrak{sl}}} % frak sl group chktex 6

% More script letters etc.
\newcommand{\SA}{\mathcal A}
\newcommand{\SB}{\mathcal B}
\newcommand{\SC}{\mathcal C}
\newcommand{\SF}{\mathcal F}
\newcommand{\SG}{\mathcal G}
\newcommand{\SH}{\mathcal H}
\newcommand{\OO}{\mathcal O}

\newcommand{\SCA}{\mathscr A}
\newcommand{\SCB}{\mathscr B}
\newcommand{\SCC}{\mathscr C}
\newcommand{\SCD}{\mathscr D}
\newcommand{\SCE}{\mathscr E}
\newcommand{\SCF}{\mathscr F}
\newcommand{\SCG}{\mathscr G}
\newcommand{\SCH}{\mathscr H}

% Mathfrak primes
\newcommand{\km}{\mathfrak m}
\newcommand{\kp}{\mathfrak p}
\newcommand{\kq}{\mathfrak q}

% number sets
\newcommand{\RR}[1][]{\ensuremath{\ifstrempty{#1}{\mathbb{R}}{\mathbb{R}^{#1}}}}
\newcommand{\NN}[1][]{\ensuremath{\ifstrempty{#1}{\mathbb{N}}{\mathbb{N}^{#1}}}}
\newcommand{\ZZ}[1][]{\ensuremath{\ifstrempty{#1}{\mathbb{Z}}{\mathbb{Z}^{#1}}}}
\newcommand{\QQ}[1][]{\ensuremath{\ifstrempty{#1}{\mathbb{Q}}{\mathbb{Q}^{#1}}}}
\newcommand{\CC}[1][]{\ensuremath{\ifstrempty{#1}{\mathbb{C}}{\mathbb{C}^{#1}}}}
\newcommand{\PP}[1][]{\ensuremath{\ifstrempty{#1}{\mathbb{P}}{\mathbb{P}^{#1}}}}
\newcommand{\HH}[1][]{\ensuremath{\ifstrempty{#1}{\mathbb{H}}{\mathbb{H}^{#1}}}}
\newcommand{\FF}[1][]{\ensuremath{\ifstrempty{#1}{\mathbb{F}}{\mathbb{F}^{#1}}}}
% expected value
\newcommand{\EE}{\ensuremath{\mathbb{E}}}
\newcommand{\charin}{\text{ char }}
\DeclareMathOperator{\sign}{sign}
\DeclareMathOperator{\Aut}{Aut}
\DeclareMathOperator{\Inn}{Inn}
\DeclareMathOperator{\Syl}{Syl}
\DeclareMathOperator{\Gal}{Gal}
\DeclareMathOperator{\GL}{GL} % General linear group
\DeclareMathOperator{\SL}{SL} % Special linear group

%---------------------------------------
% BlackBoard Math Fonts :-
%---------------------------------------

%Captital Letters
\newcommand{\bbA}{\mathbb{A}}	\newcommand{\bbB}{\mathbb{B}}
\newcommand{\bbC}{\mathbb{C}}	\newcommand{\bbD}{\mathbb{D}}
\newcommand{\bbE}{\mathbb{E}}	\newcommand{\bbF}{\mathbb{F}}
\newcommand{\bbG}{\mathbb{G}}	\newcommand{\bbH}{\mathbb{H}}
\newcommand{\bbI}{\mathbb{I}}	\newcommand{\bbJ}{\mathbb{J}}
\newcommand{\bbK}{\mathbb{K}}	\newcommand{\bbL}{\mathbb{L}}
\newcommand{\bbM}{\mathbb{M}}	\newcommand{\bbN}{\mathbb{N}}
\newcommand{\bbO}{\mathbb{O}}	\newcommand{\bbP}{\mathbb{P}}
\newcommand{\bbQ}{\mathbb{Q}}	\newcommand{\bbR}{\mathbb{R}}
\newcommand{\bbS}{\mathbb{S}}	\newcommand{\bbT}{\mathbb{T}}
\newcommand{\bbU}{\mathbb{U}}	\newcommand{\bbV}{\mathbb{V}}
\newcommand{\bbW}{\mathbb{W}}	\newcommand{\bbX}{\mathbb{X}}
\newcommand{\bbY}{\mathbb{Y}}	\newcommand{\bbZ}{\mathbb{Z}}

%---------------------------------------
% MathCal Fonts :-
%---------------------------------------

%Captital Letters
\newcommand{\mcA}{\mathcal{A}}	\newcommand{\mcB}{\mathcal{B}}
\newcommand{\mcC}{\mathcal{C}}	\newcommand{\mcD}{\mathcal{D}}
\newcommand{\mcE}{\mathcal{E}}	\newcommand{\mcF}{\mathcal{F}}
\newcommand{\mcG}{\mathcal{G}}	\newcommand{\mcH}{\mathcal{H}}
\newcommand{\mcI}{\mathcal{I}}	\newcommand{\mcJ}{\mathcal{J}}
\newcommand{\mcK}{\mathcal{K}}	\newcommand{\mcL}{\mathcal{L}}
\newcommand{\mcM}{\mathcal{M}}	\newcommand{\mcN}{\mathcal{N}}
\newcommand{\mcO}{\mathcal{O}}	\newcommand{\mcP}{\mathcal{P}}
\newcommand{\mcQ}{\mathcal{Q}}	\newcommand{\mcR}{\mathcal{R}}
\newcommand{\mcS}{\mathcal{S}}	\newcommand{\mcT}{\mathcal{T}}
\newcommand{\mcU}{\mathcal{U}}	\newcommand{\mcV}{\mathcal{V}}
\newcommand{\mcW}{\mathcal{W}}	\newcommand{\mcX}{\mathcal{X}}
\newcommand{\mcY}{\mathcal{Y}}	\newcommand{\mcZ}{\mathcal{Z}}


%---------------------------------------
% Bold Math Fonts :-
%---------------------------------------

%Captital Letters
\newcommand{\bmA}{\boldsymbol{A}}	\newcommand{\bmB}{\boldsymbol{B}}
\newcommand{\bmC}{\boldsymbol{C}}	\newcommand{\bmD}{\boldsymbol{D}}
\newcommand{\bmE}{\boldsymbol{E}}	\newcommand{\bmF}{\boldsymbol{F}}
\newcommand{\bmG}{\boldsymbol{G}}	\newcommand{\bmH}{\boldsymbol{H}}
\newcommand{\bmI}{\boldsymbol{I}}	\newcommand{\bmJ}{\boldsymbol{J}}
\newcommand{\bmK}{\boldsymbol{K}}	\newcommand{\bmL}{\boldsymbol{L}}
\newcommand{\bmM}{\boldsymbol{M}}	\newcommand{\bmN}{\boldsymbol{N}}
\newcommand{\bmO}{\boldsymbol{O}}	\newcommand{\bmP}{\boldsymbol{P}}
\newcommand{\bmQ}{\boldsymbol{Q}}	\newcommand{\bmR}{\boldsymbol{R}}
\newcommand{\bmS}{\boldsymbol{S}}	\newcommand{\bmT}{\boldsymbol{T}}
\newcommand{\bmU}{\boldsymbol{U}}	\newcommand{\bmV}{\boldsymbol{V}}
\newcommand{\bmW}{\boldsymbol{W}}	\newcommand{\bmX}{\boldsymbol{X}}
\newcommand{\bmY}{\boldsymbol{Y}}	\newcommand{\bmZ}{\boldsymbol{Z}}
%Small Letters
\newcommand{\bma}{\boldsymbol{a}}	\newcommand{\bmb}{\boldsymbol{b}}
\newcommand{\bmc}{\boldsymbol{c}}	\newcommand{\bmd}{\boldsymbol{d}}
\newcommand{\bme}{\boldsymbol{e}}	\newcommand{\bmf}{\boldsymbol{f}}
\newcommand{\bmg}{\boldsymbol{g}}	\newcommand{\bmh}{\boldsymbol{h}}
\newcommand{\bmi}{\boldsymbol{i}}	\newcommand{\bmj}{\boldsymbol{j}}
\newcommand{\bmk}{\boldsymbol{k}}	\newcommand{\bml}{\boldsymbol{l}}
\newcommand{\bmm}{\boldsymbol{m}}	\newcommand{\bmn}{\boldsymbol{n}}
\newcommand{\bmo}{\boldsymbol{o}}	\newcommand{\bmp}{\boldsymbol{p}}
\newcommand{\bmq}{\boldsymbol{q}}	\newcommand{\bmr}{\boldsymbol{r}}
\newcommand{\bms}{\boldsymbol{s}}	\newcommand{\bmt}{\boldsymbol{t}}
\newcommand{\bmu}{\boldsymbol{u}}	\newcommand{\bmv}{\boldsymbol{v}}
\newcommand{\bmw}{\boldsymbol{w}}	\newcommand{\bmx}{\boldsymbol{x}}
\newcommand{\bmy}{\boldsymbol{y}}	\newcommand{\bmz}{\boldsymbol{z}}

%---------------------------------------
% Scr Math Fonts :-
%---------------------------------------

\newcommand{\sA}{{\mathscr{A}}}   \newcommand{\sB}{{\mathscr{B}}}
\newcommand{\sC}{{\mathscr{C}}}   \newcommand{\sD}{{\mathscr{D}}}
\newcommand{\sE}{{\mathscr{E}}}   \newcommand{\sF}{{\mathscr{F}}}
\newcommand{\sG}{{\mathscr{G}}}   \newcommand{\sH}{{\mathscr{H}}}
\newcommand{\sI}{{\mathscr{I}}}   \newcommand{\sJ}{{\mathscr{J}}}
\newcommand{\sK}{{\mathscr{K}}}   \newcommand{\sL}{{\mathscr{L}}}
\newcommand{\sM}{{\mathscr{M}}}   \newcommand{\sN}{{\mathscr{N}}}
\newcommand{\sO}{{\mathscr{O}}}   \newcommand{\sP}{{\mathscr{P}}}
\newcommand{\sQ}{{\mathscr{Q}}}   \newcommand{\sR}{{\mathscr{R}}}
\newcommand{\sS}{{\mathscr{S}}}   \newcommand{\sT}{{\mathscr{T}}}
\newcommand{\sU}{{\mathscr{U}}}   \newcommand{\sV}{{\mathscr{V}}}
\newcommand{\sW}{{\mathscr{W}}}   \newcommand{\sX}{{\mathscr{X}}}
\newcommand{\sY}{{\mathscr{Y}}}   \newcommand{\sZ}{{\mathscr{Z}}}


%---------------------------------------
% Math Fraktur Font
%---------------------------------------

%Captital Letters
\newcommand{\mfA}{\mathfrak{A}}	\newcommand{\mfB}{\mathfrak{B}}
\newcommand{\mfC}{\mathfrak{C}}	\newcommand{\mfD}{\mathfrak{D}}
\newcommand{\mfE}{\mathfrak{E}}	\newcommand{\mfF}{\mathfrak{F}}
\newcommand{\mfG}{\mathfrak{G}}	\newcommand{\mfH}{\mathfrak{H}}
\newcommand{\mfI}{\mathfrak{I}}	\newcommand{\mfJ}{\mathfrak{J}}
\newcommand{\mfK}{\mathfrak{K}}	\newcommand{\mfL}{\mathfrak{L}}
\newcommand{\mfM}{\mathfrak{M}}	\newcommand{\mfN}{\mathfrak{N}}
\newcommand{\mfO}{\mathfrak{O}}	\newcommand{\mfP}{\mathfrak{P}}
\newcommand{\mfQ}{\mathfrak{Q}}	\newcommand{\mfR}{\mathfrak{R}}
\newcommand{\mfS}{\mathfrak{S}}	\newcommand{\mfT}{\mathfrak{T}}
\newcommand{\mfU}{\mathfrak{U}}	\newcommand{\mfV}{\mathfrak{V}}
\newcommand{\mfW}{\mathfrak{W}}	\newcommand{\mfX}{\mathfrak{X}}
\newcommand{\mfY}{\mathfrak{Y}}	\newcommand{\mfZ}{\mathfrak{Z}}
%Small Letters
\newcommand{\mfa}{\mathfrak{a}}	\newcommand{\mfb}{\mathfrak{b}}
\newcommand{\mfc}{\mathfrak{c}}	\newcommand{\mfd}{\mathfrak{d}}
\newcommand{\mfe}{\mathfrak{e}}	\newcommand{\mff}{\mathfrak{f}}
\newcommand{\mfg}{\mathfrak{g}}	\newcommand{\mfh}{\mathfrak{h}}
\newcommand{\mfi}{\mathfrak{i}}	\newcommand{\mfj}{\mathfrak{j}}
\newcommand{\mfk}{\mathfrak{k}}	\newcommand{\mfl}{\mathfrak{l}}
\newcommand{\mfm}{\mathfrak{m}}	\newcommand{\mfn}{\mathfrak{n}}
\newcommand{\mfo}{\mathfrak{o}}	\newcommand{\mfp}{\mathfrak{p}}
\newcommand{\mfq}{\mathfrak{q}}	\newcommand{\mfr}{\mathfrak{r}}
\newcommand{\mfs}{\mathfrak{s}}	\newcommand{\mft}{\mathfrak{t}}
\newcommand{\mfu}{\mathfrak{u}}	\newcommand{\mfv}{\mathfrak{v}}
\newcommand{\mfw}{\mathfrak{w}}	\newcommand{\mfx}{\mathfrak{x}}
\newcommand{\mfy}{\mathfrak{y}}	\newcommand{\mfz}{\mathfrak{z}}


\usepackage{appendixnumberbeamer}
\usepackage{tikz}
\usepackage{tikz-cd}
\usepackage{graphicx}
\usetikzlibrary{positioning,quotes,calc,patterns,babel}
\usetikzlibrary{shapes.geometric,calc, decorations.pathmorphing, arrows.meta, overlay-beamer-styles}
\usetheme[progressbar=frametitle, block=fill]{metropolis}

\title{Cluster algebras and where to find them}
\date{June 28, 2024}
\author{Wannes Malfait}
\institute{Vrije Universiteit Brussel}


\begin{document}
\maketitle
\section{Combinatorics $\to$ algebra}
\begin{frame}{Triangulations}
	\uncover<2>{
		\begin{center}
			Triangulations
		\end{center}
	}
	\begin{center}
		\begin{tikzpicture}
			\node[name=h,shape=regular polygon, regular polygon sides = 6, draw, inner sep = 10ex] {};
			\only<2>{
				\draw (h.corner 1) -- (h.corner 3);
				\draw (h.corner 1) -- (h.corner 4);
				\draw (h.corner 1) -- (h.corner 5);
			}
		\end{tikzpicture}
	\end{center}
\end{frame}

\begin{frame}
	\frametitle{Flipping diagonals}
	\begin{columns}[c]
		\begin{column}{0.4\textwidth}
			Observations:
			\begin{itemize}
				\item Flipping a diagonal
				      %
				\item<5-> Ptolemy's rule\uncover<6->{:
						\begin{align*}
							{\color{red} |ac| \cdot |b d|} = |ab| \cdot |cd| + |ad| \cdot |bc| \\
							\leadsto {\color{red} |bd|} = \frac{|ab| \cdot |cd| + |ad| \cdot |bc|}{\color{red} |ac|}
						\end{align*}}
			\end{itemize}
		\end{column}
		\begin{column}{0.6\textwidth}
			\begin{center}
				\begin{tikzpicture}
					\node[name=h,shape=regular polygon, regular polygon sides = 6, draw, alt=<{2,3,5,6}>{gray}{mDarkTeal}, minimum size = 30ex] {};
					\draw (h.corner 1) -- (h.corner 3);
					\only<1,2,5,6> {
						\draw[alt=<{2,5,6}>{dashed, red}{solid, mDarkTeal}] (h.corner 1) -- (h.corner 4);
					}
					\only<3->{
						\draw[alt=<{3,5,6}>{dashed, red}{solid, mDarkTeal}] (h.corner 3) -- (h.corner 5);
					}
					\uncover<6>{
						\node[circle, draw, minimum size = 30ex] {};
					}
					\only<{2,3,5,6}>{
						\draw (h.corner 3) -- (h.corner 4);
						\draw (h.corner 4) -- (h.corner 5);
					}
					\draw (h.corner 1) -- (h.corner 5);
					\uncover<{5,6}>{
						\node[above] at (h.corner 1) {a};
						\node[left] at (h.corner 3) {b};
						\node[below] at (h.corner 4) {c};
						\node[below] at (h.corner 5) {d};
					}
				\end{tikzpicture}
			\end{center}
		\end{column}
	\end{columns}
\end{frame}

\begin{frame}
	\frametitle{The cluster algebra}

	\begin{equation*}
		\{\text{ sides }\} \cup \{ \text{ diagonals }\} \longrightarrow \{x_1, \dotsc, x_m\}
	\end{equation*}
	\vspace{-1em}
	\begin{itemize}
		\item $\{x_1, \dotsc, x_m\}$ is a \emph{cluster} with \emph{cluster variables} $x_1, \dotsc, x_m$\pause
		\item Flip $\leadsto$ mutate a variable\pause
		      \begin{equation*}
			      {\color{red} x_i'} = \frac{x_a x_c + x_b x_d}{\color{red} x_i}
		      \end{equation*}
		      \begin{equation*}
			      \begin{tikzpicture}[baseline]
				      \node[name=r, draw, inner sep = 3.5ex, xscale = 1.5]     {};
				      \node[above] at  (r.north) {$x_a$};
				      \node[left]  at  (r.west) {$x_b$};
				      \node[below] at  (r.south) {$x_c$};
				      \node[right] at  (r.east) {$x_d$};
				      \draw[red, dashed] (r.south west) edge["$x_i$"] (r.north east);
			      \end{tikzpicture}
			      \qquad\longrightarrow\qquad
			      \begin{tikzpicture}[baseline]
				      \node[name=r, draw, inner sep = 3.5ex, xscale = 1.5]        {};
				      \node[above] at  (r.north) {$x_a$};
				      \node[left]  at  (r.west) {$x_b$};
				      \node[below] at  (r.south) {$x_c$};
				      \node[right] at  (r.east) {$x_d$};
				      \draw[red, dashed] (r.north west) edge["$x_i'$"] (r.south east);
			      \end{tikzpicture}
		      \end{equation*}
		      \pause
		\item \emph{Cluster algebra} $=$ subalgebra of $\mathbb{C}(x_1, \dotsc, x_m)$ generated by cluster variables.
	\end{itemize}

\end{frame}

\begin{frame}
	\frametitle{Exchange graph}
	\begin{center}

		\resizebox{!}{0.95\textheight}{%
			\begin{tikzpicture}[
					every node/.style={draw, minimum size = 5ex, shape=regular polygon, regular polygon sides =6},
				]
				\def\radius{20ex}
				\begin{scope}[rotate=90]
					\node[name=H1] at (0:\radius) {};
					\draw (H1.corner 3) -- (H1.corner 5);
					\draw (H1.corner 1) -- (H1.corner 3);
					\draw (H1.corner 1) -- (H1.corner 5);
					\node[name=H2] at (72:\radius) {};
					\draw (H2.corner 1) -- (H2.corner 3);
					\draw (H2.corner 1) -- (H2.corner 4);
					\draw (H2.corner 1) -- (H2.corner 5);
					\node[name=H3] at (144:\radius) {};
					\draw (H3.corner 2) -- (H3.corner 4);
					\draw (H3.corner 1) -- (H3.corner 4);
					\draw (H3.corner 1) -- (H3.corner 5);
					\node[name=H4] at (216:\radius) {};
					\draw (H4.corner 2) -- (H4.corner 4);
					\draw (H4.corner 2) -- (H4.corner 5);
					\draw (H4.corner 1) -- (H4.corner 5);
					\node[name=H5] at (288:\radius) {};
					\draw (H5.corner 3) -- (H5.corner 5);
					\draw (H5.corner 2) -- (H5.corner 5);
					\draw (H5.corner 1) -- (H5.corner 5);

					\node[name=H6, below=30ex of H2] {};
					\node[name=H7, below right=12ex and 5ex of H5] {};
					\node[name=H8, below left=12ex and 5ex of H2] {};
					\node[name=H9, below=5ex of H1] {};
					\node[name=H10, below=27ex of H9] {};
					\node[name=H11, below right=10ex and 5ex of H9] {};
					\node[name=H12, below left=10ex and 5ex of H9] {};
					\node[name=H13, below=13ex of H10] {};
					\node[name=H14, below=30ex of H5] {};

					\draw (H13.corner 2) -- (H13.corner 4);
					\draw (H13.corner 6) -- (H13.corner 4);
					\draw (H13.corner 6) -- (H13.corner 2);

					\draw (H14.corner 2) -- (H14.corner 4);
					\draw (H14.corner 2) -- (H14.corner 6);
					\draw (H14.corner 2) -- (H14.corner 5);

					\draw (H6.corner 2) -- (H6.corner 4);
					\draw (H6.corner 1) -- (H6.corner 4);
					\draw (H6.corner 6) -- (H6.corner 4);

					\draw (H8.corner 1) -- (H8.corner 3);
					\draw (H8.corner 1) -- (H8.corner 4);
					\draw (H8.corner 6) -- (H8.corner 4);

					\draw (H7.corner 3) -- (H7.corner 5);
					\draw (H7.corner 2) -- (H7.corner 6);
					\draw (H7.corner 2) -- (H7.corner 5);

					\draw (H9.corner 3) -- (H9.corner 5);
					\draw (H9.corner 1) -- (H9.corner 3);
					\draw (H9.corner 3) -- (H9.corner 6);

					\draw (H11.corner 3) -- (H11.corner 5);
					\draw (H11.corner 2) -- (H11.corner 6);
					\draw (H11.corner 3) -- (H11.corner 6);

					\draw (H12.corner 4) -- (H12.corner 6);
					\draw (H12.corner 1) -- (H12.corner 3);
					\draw (H12.corner 3) -- (H12.corner 6);

					\draw (H10.corner 3) -- (H10.corner 6);
					\draw (H10.corner 6) -- (H10.corner 4);
					\draw (H10.corner 6) -- (H10.corner 2);

					% Fake nodes to have some padding. 
					% Can't set outer sep on original nodes because it messes up the diagonals.
					\begin{scope}[nodes={draw=none, outer sep = 1ex}]
						\foreach \x in {1,...,14} {\node[name=H\x'] at (H\x) {};}
					\end{scope}

					\begin{scope}[every path/.style={densely dashed}]
						\draw (H1') -- (H2') -- (H3') -- (H4') -- (H5') -- (H1');
						\draw (H2') -- (H8') -- (H6') -- (H13') -- (H14') -- (H7') -- (H5');
						\draw (H9') -- (H11') -- (H10') -- (H12') -- (H9');
						\draw (H9') -- (H1');
						\draw (H8') -- (H12');
						\draw (H7') -- (H11');
						\draw (H6') -- (H3');
						\draw (H14') -- (H4');
						\draw (H13') -- (H10');
					\end{scope}
				\end{scope}
			\end{tikzpicture}
		}%
	\end{center}
\end{frame}

\begin{frame}
	\frametitle{The quiver}

	\begin{center}
		\begin{tikzpicture}
			\node[outer sep=1ex] (triangulations) at (-3.5,0) {\{Triangulations of $n$-gon\}};
			\node[outer sep=1ex] (clusteralgebra) at (4, 0) {Commutative algebra};
			\draw[->, alt=<1>{solid}{dashed}] (triangulations) -- (clusteralgebra);
			\uncover<2->{
				\node[outer sep=1ex] (graph) at (0,-1.5) {\{Quivers\}};
				\draw[->] (triangulations) -- (graph);
				\draw[->] (graph) -- (clusteralgebra);
			}
		\end{tikzpicture}
	\end{center}
	\uncover<3->{
		\begin{center}
			\begin{tikzpicture}[
					R/.style = {rectangle, alt=<{3,6}>{fill=none}{fill=mLightBrown}, inner sep=0.5ex, outer sep=0.5ex},
					B/.style = {circle, alt=<{3,6}>{fill=none}{fill=teal}, inner sep=0.5ex, outer sep=0.5ex},
					every edge/.style= {draw, alt=<{5}>{gray}{mDarkTeal}},
					baseline
				]
				\coordinate (center) at (0,0);
				\def\radius{10ex}

				% points on the circle
				\begin{scope}[rotate=120]
					\path (center) ++(0:\radius) coordinate (A);
					\path (center) ++(60:\radius) coordinate (B);
					\path (center) ++(120:\radius) coordinate (C);
					\path (center) ++(180:\radius) coordinate (D);
					\path (center) ++(240:\radius) coordinate (E);
					\path (center) ++(300:\radius) coordinate (F);
				\end{scope}

				% Sides with frozen verts
				\begin{scope}[nodes=R]
					\path (A) edge node (AB) {} (B);
					\path (B) edge node (BC) {} (C);
					\path (C) edge node (CD) {} (D);
					\path (D) edge node (DE) {} (E);
					\path (E) edge node (EF) {} (F);
					\path (F) edge node (AF) {} (A);
				\end{scope}

				% Diagonals with exchangeable verts
				\begin{scope}[nodes=B]
					\path (A) edge node (AC) {} (C);
					\path (A) edge node (AD) {} (D);
					\path (D) edge node (DF) {} (F);
				\end{scope}

				% Quiver edges
				\begin{scope}[every edge/.style={alt=<{3,4,6}>{draw=none}{draw}, -Latex,mDarkTeal,line width=0.2ex}]
					\path (AB) edge (AC);
					\path (AC) edge (BC);
					\path (AC) edge (AD);
					\path (CD) edge (AC);
					\path (AD) edge (CD);
					\path (AD) edge (AF);
					\path (AF) edge (DF);
					\path (DF) edge (AD);
					\path (DF) edge (EF);
					\path (DE) edge (DF);
				\end{scope}
			\end{tikzpicture}
			\only<6>{
				$\qquad \longrightarrow \qquad$
				\begin{tikzpicture}[
						R/.style = {rectangle, fill=mLightBrown, inner sep=0.5ex, outer sep= 0.5ex},
						B/.style = {circle, fill=teal, inner sep=0.5ex, outer sep= 0.5ex},
						every edge/.style= {draw=none},
						baseline
					]
					\coordinate (center) at (0,0);
					\def\radius{10ex}

					% points on the circle
					\begin{scope}[rotate=120]
						\path (center) ++(0:\radius) coordinate (A);
						\path (center) ++(60:\radius) coordinate (B);
						\path (center) ++(120:\radius) coordinate (C);
						\path (center) ++(180:\radius) coordinate (D);
						\path (center) ++(240:\radius) coordinate (E);
						\path (center) ++(300:\radius) coordinate (F);
					\end{scope}

					% Sides with frozen verts
					\begin{scope}[nodes=R]
						\path (A) edge node (AB) {} (B);
						\path (B) edge node (BC) {} (C);
						\path (C) edge node (CD) {} (D);
						\path (D) edge node (DE) {} (E);
						\path (E) edge node (EF) {} (F);
						\path (F) edge node (AF) {} (A);
					\end{scope}

					% Diagonals with exchangeable verts
					\begin{scope}[nodes=B]
						\path (A) edge node (AC) {} (C);
						\path (A) edge node (AD) {} (D);
						\path (D) edge node (DF) {} (F);
					\end{scope}

					% Quiver edges
					\begin{scope}[every edge/.style={draw, -Latex,mDarkTeal,line width=0.2ex}]
						\path (AB) edge (AC);
						\path (AC) edge (BC);
						\path (AC) edge (AD);
						\path (CD) edge (AC);
						\path (AD) edge (CD);
						\path (AD) edge (AF);
						\path (AF) edge (DF);
						\path (DF) edge (AD);
						\path (DF) edge (EF);
						\path (DE) edge (DF);
					\end{scope}
				\end{tikzpicture}
			}
		\end{center}
	}
\end{frame}

\begin{frame}
	\frametitle{Quiver Mutation}
	\begin{columns}[c]

		\begin{column}{0.4\textwidth}
			\begin{center}

				\begin{tikzpicture}[
						R/.style = {rectangle, fill=none, inner sep=0.7ex},
						B/.style = {circle, fill=none, inner sep=0.7ex},
						every edge/.style= {draw, mDarkTeal},
						baseline
					]
					\coordinate (center) at (0,0);
					\def\radius{8ex}

					% points on the circle
					\begin{scope}[rotate=120]
						\path (center) ++(0:\radius) coordinate (A);
						\path (center) ++(60:\radius) coordinate (B);
						\path (center) ++(120:\radius) coordinate (C);
						\path (center) ++(180:\radius) coordinate (D);
						\path (center) ++(240:\radius) coordinate (E);
						\path (center) ++(300:\radius) coordinate (F);
					\end{scope}

					% Sides with frozen verts
					\begin{scope}[nodes=R]
						\path (A) edge node (AB) {} (B);
						\path (B) edge node (BC) {} (C);
						\path (C) edge node (CD) {} (D);
						\path (D) edge node (DE) {} (E);
						\path (E) edge node (EF) {} (F);
						\path (F) edge node (AF) {} (A);
					\end{scope}

					% Diagonals with exchangeable verts
					\begin{scope}[nodes=B]
						\path[dashed] (A) edge[red] node (AC) {} (C);
						\path (A) edge node (AD) {} (D);
						\path (D) edge node (DF) {} (F);
					\end{scope}
				\end{tikzpicture}
				\uncover<2->{
					\begin{equation*}
						\downarrow
					\end{equation*}

					\begin{tikzpicture}[
							R/.style = {rectangle, fill=none, inner sep=0.7ex},
							B/.style = {circle, fill=none, inner sep=0.7ex},
							every edge/.style= {draw, mDarkTeal},
							baseline
						]
						\coordinate (center) at (0,0);
						\def\radius{8ex}

						% points on the circle
						\begin{scope}[rotate=120]
							\path (center) ++(0:\radius) coordinate (A);
							\path (center) ++(60:\radius) coordinate (B);
							\path (center) ++(120:\radius) coordinate (C);
							\path (center) ++(180:\radius) coordinate (D);
							\path (center) ++(240:\radius) coordinate (E);
							\path (center) ++(300:\radius) coordinate (F);
						\end{scope}

						% Sides with frozen verts
						\begin{scope}[nodes=R]
							\path (A) edge node (AB) {} (B);
							\path (B) edge node (BC) {} (C);
							\path (C) edge node (CD) {} (D);
							\path (D) edge node (DE) {} (E);
							\path (E) edge node (EF) {} (F);
							\path (F) edge node (AF) {} (A);
						\end{scope}

						% Diagonals with exchangeable verts
						\begin{scope}[nodes=B]
							\path (A) edge node (AD) {} (D);
							\path[dashed] (B) edge[red] node (BD) {} (D);
							\path (D) edge node (DF) {} (F);
							% \path (A) edge node (AG) {} (G);
							% \path (D) edge node (DG) {} (G);
							% \path (E) edge node (EG) {} (G);
						\end{scope}
					\end{tikzpicture}
				}
			\end{center}
		\end{column}

		\begin{column}{0.2\textwidth}
			\uncover<3>{
				\begin{equation*}
					x_i' = \frac{1}{x_i}\Bigl( \prod_{i \to j} x_j + \prod_{j \to i} x_j\Bigr)
				\end{equation*}
			}
		\end{column}

		\begin{column}{0.4\textwidth}
			\begin{center}

				\begin{tikzpicture}[
						R/.style = {rectangle, fill=mLightBrown, inner sep=0.5ex, outer sep=.5ex},
						B/.style = {circle, fill=teal, inner sep=0.5ex, outer sep=0.5ex},
						every edge/.style= {draw=none},
						baseline
					]
					\coordinate (center) at (0,0);
					\def\radius{9ex}

					% points on the circle
					\begin{scope}[rotate=120]
						\path (center) ++(0:\radius) coordinate (A);
						\path (center) ++(60:\radius) coordinate (B);
						\path (center) ++(120:\radius) coordinate (C);
						\path (center) ++(180:\radius) coordinate (D);
						\path (center) ++(240:\radius) coordinate (E);
						\path (center) ++(300:\radius) coordinate (F);
					\end{scope}

					% Sides with frozen verts
					\begin{scope}[nodes=R]
						\path (A) edge node (AB) {} (B);
						\path (B) edge node (BC) {} (C);
						\path (C) edge node (CD) {} (D);
						\path (D) edge node (DE) {} (E);
						\path (E) edge node (EF) {} (F);
						\path (F) edge node (AF) {} (A);
					\end{scope}

					% Diagonals with exchangeable verts
					\begin{scope}[nodes=B]
						\path (A) edge node[red] (AC) {} (C);
						\path (A) edge node (AD) {} (D);
						\path (D) edge node (DF) {} (F);
					\end{scope}

					% Quiver edges
					\begin{scope}[every edge/.style={draw, -Latex,mDarkTeal,line width=0.15ex}]
						\path (AB) edge (AC);
						\path (AC) edge (BC);
						\path (AC) edge (AD);
						\path (CD) edge (AC);
						\path (AD) edge (CD);
						\path (AD) edge (AF);
						\path (AF) edge (DF);
						\path (DF) edge (AD);
						\path (DF) edge (EF);
						\path (DE) edge (DF);
					\end{scope}
				\end{tikzpicture}

				\uncover<2->{

					\begin{equation*}
						\downarrow
					\end{equation*}

					\begin{tikzpicture}[
							R/.style = {rectangle, fill=mLightBrown, inner sep=0.5ex, outer sep = 0.5ex},
							B/.style = {circle, fill=teal, inner sep=0.5ex, outer sep = 0.5ex},
							every edge/.style= {draw=none},
							baseline
						]
						\coordinate (center) at (0,0);
						\def\radius{8ex}

						% points on the circle
						\begin{scope}[rotate=120]
							\path (center) ++(0:\radius) coordinate (A);
							\path (center) ++(60:\radius) coordinate (B);
							\path (center) ++(120:\radius) coordinate (C);
							\path (center) ++(180:\radius) coordinate (D);
							\path (center) ++(240:\radius) coordinate (E);
							\path (center) ++(300:\radius) coordinate (F);
						\end{scope}

						% Sides with frozen verts
						\begin{scope}[nodes=R]
							\path (A) edge node (AB) {} (B);
							\path (B) edge node (BC) {} (C);
							\path (C) edge node (CD) {} (D);
							\path (D) edge node (DE) {} (E);
							\path (E) edge node (EF) {} (F);
							\path (F) edge node (AF) {} (A);
						\end{scope}

						% Diagonals with exchangeable verts
						\begin{scope}[nodes=B]
							\path (A) edge node (AD) {} (D);
							\path (B) edge node[fill=red] (BD) {} (D);
							\path (D) edge node (DF) {} (F);
						\end{scope}

						% Quiver edges
						\begin{scope}[every edge/.style={draw, -Latex,mDarkTeal,line width=0.15ex}]
							\path (BD) edge[red] (AB);
							\path (AB) edge[red] (AD);
							\path (AD) edge[red] (BD);

							\path (BC) edge[red] (BD);
							\path (BD) edge[red] (CD);

							\path (AD) edge (AF);
							\path (AF) edge (DF);
							\path (DF) edge (AD);

							\path (DF) edge (EF);
							\path (DE) edge (DF);
						\end{scope}
					\end{tikzpicture}
				}
			\end{center}
		\end{column}

	\end{columns}
\end{frame}

\begin{frame}
	\frametitle{Hexagon is a surface}

	\begin{center}
		\begin{tikzpicture}
			\node[draw, shape=regular polygon, regular polygon sides=6, minimum size=20ex](hex){};
			\pause
			\node[draw, shape=circle, fill=mLightBrown!30, right= 10ex of hex, minimum size=20ex] (disc) {};
			\path (hex) -- node[font=\Large] {=} (disc);
			\foreach \a in {0, 60, ..., 360} {
					\node[fill, inner sep=0.3ex, shape=circle] at ($(\a :10ex) + (disc)$) {};
				}

		\end{tikzpicture}
	\end{center}

\end{frame}

\begin{frame}
	\frametitle{Another example}

	\begin{center}
		\begin{tikzpicture}[nodes={shape=circle,line width = 0.2ex, draw,font=\footnotesize},baseline]
			\fill[fill=mLightBrown!30, even odd rule] (0,0) circle[radius=18ex] (0,0) circle[radius=5ex];
			\node[minimum size=36ex] at (0,0) (an) {};
			\node[minimum size =10 ex] at (an.center) (hole) {};
			\node[fill, inner sep=0.3ex] at (an.south) {};
			\node[fill, inner sep=0.3ex] at (hole.west) {};
			\node[fill, inner sep=0.3ex] at (hole.east) {};

			\uncover<2->{
				\draw (an.south) edge[bend left, line width = 0.2ex] node[draw, fill=white, left, outer sep=0.5ex] {1} (hole.west);
				\draw (an.south) edge[bend right, line width = 0.2ex] node[draw,fill=white, left, near start, outer sep=1ex] {2} (hole.east);
				\draw[line width = 0.2ex] plot [smooth, tension=1] coordinates {
						(an.south)
						($(hole.east) + (1.5,0)$) ($(hole.north) + (0,1.5)$)
						(hole.west)
					};
				\node[draw,fill=white, left, outer sep=5ex] at (an.east) {3};

				\node[fill=white, left, rectangle, outer sep=0.3ex] at (an.west) {$a$};
				\node[fill=white, below, rectangle, outer sep=0.3ex] at (hole.south) {$b$};
				\node[fill=white, above, rectangle, outer sep=0.3ex] at (hole.north) {$c$};
			}
		\end{tikzpicture}
		\uncover<3>{
			$\quad \longrightarrow \quad$
			\begin{tikzcd}[column sep=small,ampersand replacement=\&, arrows={-Latex, line width = 0.2ex}]
				\boxed{a} \ar[rr, -Latex, line width = 0.2ex] \& \& 3 \drar \dlar \& \& \boxed{c} \ar[ll] \\
				\& 1 \ular \drar \& \& 2 \ar[ll] \urar \& \\
				\& \& \boxed{b} \urar \& \&
			\end{tikzcd}
		}
	\end{center}
\end{frame}

\section{Algebra $\to$ combinatorics}

\begin{frame}
	\frametitle{Other objects with cluster structure}

	\begin{itemize}
		\item $X$ affine variety = set of zeros of a system of polynomials
		\item e.g., $X$ is a line, a circle, a sphere, a plane, ... \pause
		\item $\mathbb{C}[X]$ coordinate algebra = polynomial functions on $X$.
	\end{itemize}
	\pause
	\textbf{Goal:} $\mathbb{C}[X] \cong$ cluster algebra
	\\
	$\leadsto$ Shows hidden combinatorial structure in $X$

\end{frame}

\begin{frame}
	\frametitle{Grassmannians}
	\begin{itemize}
		\item The Grassmannian $\Gr(d,n)$ = set of $d$-dimensional subspaces of $\bbC^n$
		\item For example, $\Gr(1, n) = \bbP(\bbC^n)$ (projective space)
	\end{itemize}
	\pause
	\begin{theorem}
		$\mathbb{C}[\wh{\Gr}(2,n)] \cong $ cluster algebra of $n$-gon.
	\end{theorem}
	\pause
	Idea of proof:
	\begin{itemize}
		\item Write down relations in $\bbC[\wh{Gr}(2,n)]$
		\item Notice they are the same as Ptolemy's rule
		\item Finish by a dimension argument
	\end{itemize}

\end{frame}

\begin{frame}
	\frametitle{Canonical bases}

	\begin{itemize}
		\item $X = G$ (simple) algebraic group, e.g., $SL_n$.
	\end{itemize}
	\pause
	\begin{theorem}[Lusztig--Kashiwara '90]
		$\mathbb{C}[G]$ has very special type of linear basis: ``dual canonical/crystal basis''.
	\end{theorem}
	\pause
	Difficulties:
	\begin{itemize}
		\item Construction
		\item Multiplication rules for basis elements
	\end{itemize}
	\pause
	$\leadsto$ Want cluster algebra structure

\end{frame}

\begin{frame}
	\frametitle{Motivation for cluster algebras}

	Cluster algebras were introduced by S. Fomin and A. Zelevinsky in 2002
	\begin{itemize}
		\item Combinatorial framework to study canonical bases and total positivity
		\item Cluster monomials form (large part of) dual canonical basis
		\item Structure constants are all positive
	\end{itemize}
\end{frame}

\section{Quantum cluster algebras}

\begin{frame}
	\frametitle{Shortcomings}

	Shortcomings:
	\begin{enumerate}
		\item Commutative algebras, so not for quantum groups\\ \pause $\leadsto$ quantum cluster
		      algebra, i.e.,
		      \begin{equation*}
			      x_i x_j = q_{ij} x_j x_i
		      \end{equation*}%
		      \pause
		\item Case-by-case
	\end{enumerate}

\end{frame}

\begin{frame}
	\frametitle{Goodearl--Yakimov}

	\begin{center}
		``Quantum cluster algebra structures on quantum nilpotent algebras''
	\end{center}
	\begin{itemize}
		\item Paper by K. Goodearl and M. Yakimov (100+ pages)
		\item Part of a series of multiple papers
		\item Applies to all ``quantum nilpotent algebras'' = CGL extensions
	\end{itemize}

\end{frame}

\begin{frame}
	\frametitle{CGL extensions}
	\begin{itemize}
		\item $A$ an algebra
		\item Ore extension $A[x; \sigma, \delta] = \{\sum_i a_i x^i\}$ is algebra such that
		      \begin{equation*}
			      x a = \sigma(a) x + \delta(a)
		      \end{equation*}
		\item CGL extension $\approx$ iterated Ore extension
		      \begin{equation*}
			      R = \bbC[x_1][x_2; \sigma_2, \delta_2][x_3; \sigma_3, \delta_3] \dotsb [x_N; \sigma_N, \delta_N]
		      \end{equation*}
	\end{itemize}
	\begin{theorem}[Goodearl--Yakimov, 2017]
		If $R$ is a CGL extension, then $R \cong$ quantum cluster algebra.
	\end{theorem}

\end{frame}

\begin{frame}
	\frametitle{Examples of CGL extensions}

	Some examples:
	\begin{itemize}
		\item $\bbC_q[M_{n\times m}]$: quantized matrix coordinate algebras
		\item $A^n_q(\bbC)$: quantized Weyl algebras
		\item More generally: $\mcU_q(\mfn_+ \cap w(\mfn_-))$ quantum Schubert cell algebras
		\item $\bbC_q[B^+ u B^- \cap B^+ v B^-]$: quantized Double Bruhat cells \pause
		\item Heisenberg double: $\HeisUqn$
	\end{itemize}
	\medskip
	\pause
	Future: integrate analytic structure of Heisenberg doubles

\end{frame}

\appendix

\section{Extra slides}

\begin{frame}
	\frametitle{Example of $\SL_2$}

	\begin{itemize}
		\item $\bbC[\SL_2] = \bbC[a,b,c,d] / (ad -bc - 1)$
		\item Rank 1 cluster algebra
		\item Frozen variables: $\{b,c\}$
		\item Exchangeable variables: $\{a,d\}$
		\item Clusters: $\{a,b,c\}$ and $\{d,b,c\}$
		\item Exchange relation
		      \begin{equation*}
			      ad = bc + 1
		      \end{equation*}
	\end{itemize}

\end{frame}
\begin{frame}
	\frametitle{$A_2$ quiver}

	\begin{align*}
		Q = \begin{tikzcd}[ampersand replacement=\&]
			    1 \rar \& 2
		    \end{tikzcd},
		 & \quad \mathbf{x} = \left(x_1, x_2\right)                                                                                  \\
		\mu_1(Q) = \begin{tikzcd}[ampersand replacement=\&]
			           1 \&  \lar 2
		           \end{tikzcd},
		 & \quad \mu_1(\mathbf{x}) = \left(\frac{1+x_2}{x_1}, x_2\right)                                                             \\
		\mu_{12}(Q) = \begin{tikzcd}[ampersand replacement=\&]
			              1 \rar \&  2
		              \end{tikzcd},
		 & \quad  \mu_{12}(\mathbf{x}) = \left(\frac{1+x_2}{x_1}, \frac{x_1 + x_2 + 1}{x_1 x_2}\right)                               \\
		\mu_{121}(Q) = \begin{tikzcd}[ampersand replacement=\&]
			               1 \& \lar 2
		               \end{tikzcd},
		 & \quad  \mu_{121}(\mathbf{x}) = \left(\frac{x_1(x_1 x_2 + x_1 + x_2 + 1)}{x_1x_2(1+x_2)}, \frac{x_1 + x_2 + 1}{x_2}\right) \\
		\mu_{1212}(Q) = \begin{tikzcd}[ampersand replacement=\&]
			                1 \rar\& 2
		                \end{tikzcd},
		 & \quad  \mu_{1212}(\mathbf{x}) = \left(\frac{x_1 + 1}{x_2}, x_1\right)                                                     \\
		\mu_{12121}(Q) = \begin{tikzcd}[ampersand replacement=\&]
			                 1 \& \lar 2
		                 \end{tikzcd},
		 & \quad  \mu_{12121}(\mathbf{x}) = \left(x_2, x_1\right)
	\end{align*}
\end{frame}
\end{document}