%% PACKAGES

\usepackage[style=alphabetic, maxnames=4]{biblatex}
\addbibresource{references.bib}

\usepackage{amsmath,amsfonts,amsthm,amssymb,mathtools}
\usepackage[shortlabels]{enumitem}
\usepackage{imakeidx}
\makeindex[intoc] % Add index to bibliography.
\usepackage[bookmarksdepth=2]{hyperref}
\usepackage{xcolor}
\hypersetup{
	colorlinks=true,
	linkcolor={red!50!black},
	citecolor={blue!50!black},
	urlcolor={blue!80!black}
}
\usepackage{cleveref}
\usepackage{tikz-cd}
\usepackage{aligned-overset}
\usepackage{microtype}

% \usepackage{fancyhdr}
% \pagestyle{fancy}

\usetikzlibrary{positioning,quotes,calc}

%% Equation numbered by section
% \numberwithin{equation}{section}

%% Theorem environments
\theoremstyle{plain}
\newtheorem{theorem}{Theorem}[section]
\newtheorem{lemma}[theorem]{Lemma}
\newtheorem{proposition}[theorem]{Proposition}
\newtheorem{corollary}[theorem]{Corollary}

\newtheorem{condition}{Condition}
\renewcommand*{\thecondition}{\Alph{condition}}
\crefname{condition}{condition}{conditions}

\newtheorem*{convention}{Convention}
\newtheorem*{theorem*}{Theorem}
\newtheorem*{lemma*}{Lemma}
\newtheorem*{prop*}{Proposition}
\newtheorem*{corollary*}{Corollary}

\theoremstyle{definition}
\newtheorem{definition}[theorem]{Definition}
\newtheorem{example}[theorem]{Example}
\newtheorem{exercise}[theorem]{Exercise}
\newtheorem*{definition*}{Definition}
\newtheorem*{example*}{Example}
\newtheorem*{exercise*}{Exercise}

\theoremstyle{remark}
\newtheorem{remark}[theorem]{Remark}
\newtheorem*{remark*}{Remark}

%===========================================================%
%The code below customises theorem numbering
%===========================================================%

\theoremstyle{plain}
\newtheorem{innercustomgenericplain}{\customgenericname}
\providecommand{\customgenericname}{}
\newcommand{\newcustomtheoremplain}[2]{%
	\crefname{#2}{#2}{#2s}%
	\newenvironment{#1}[1]
	{%
		\renewcommand\customgenericname{#2}%
		\crefalias{innercustomgenericplain}{#2}%
		\renewcommand\theinnercustomgenericplain{##1}%
		\innercustomgenericplain
	}
	{\endinnercustomgenericplain}
}

\theoremstyle{definition}
\newtheorem{innercustomgenericdefinition}{\customgenericname}
\providecommand{\customgenericname}{}
\newcommand{\newcustomtheoremdefinition}[2]{%
	\crefname{#2}{#2}{#2s}%
	\newenvironment{#1}[1]
	{%
		\renewcommand\customgenericname{#2}%
		\crefalias{innercustomgenericdefinition}{#2}%
		\renewcommand\theinnercustomgenericdefinition{##1}%
		\innercustomgenericdefinition
	}
	{\endinnercustomgenericdefinition}
}

\theoremstyle{remark}
\newtheorem{innercustomgenericremark}{\customgenericname}
\providecommand{\customgenericname}{}
\newcommand{\newcustomtheoremremark}[2]{%
	\crefname{#2}{#2}{#2s}%
	\newenvironment{#1}[1]
	{%
		\renewcommand\customgenericname{#2}%
		\crefalias{innercustomgenericremark}{#2}%
		\renewcommand\theinnercustomgenericremark{##1}%
		\innercustomgenericremark
	}
	{\endinnercustomgenericremark}
}

\newcustomtheoremplain{customtheorem}{Theorem}
\newcustomtheoremplain{customlemma}{Lemma}
\newcustomtheoremplain{customprop}{Proposition}
\newcustomtheoremplain{customcorollary}{Corollary}

\newcustomtheoremdefinition{customdefinition}{Definition}
\newcustomtheoremdefinition{customexample}{Example}
\newcustomtheoremdefinition{customexercise}{Exercise}

\newcustomtheoremremark{customremark}{Remark}
