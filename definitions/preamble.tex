%% PACKAGES

% Fancy chapters with toc
\usepackage{titlesec,titletoc}

\usepackage[style=alphabetic, maxnames=4]{biblatex}
\addbibresource{references.bib}

\usepackage{amsmath,amsfonts,amsthm,amssymb,mathtools}
\usepackage[shortlabels]{enumitem}
\usepackage{imakeidx}
\makeindex[intoc] % Add index to bibliography.
\usepackage[bookmarksdepth=2]{hyperref}
\usepackage{xcolor}
\hypersetup{
	colorlinks=true,
	linkcolor={red!50!black},
	citecolor={blue!50!black},
	urlcolor={blue!80!black}
}
\usepackage{cleveref}
\usepackage{tikz-cd}
\usepackage{aligned-overset}
\usepackage{tcolorbox}
\tcbuselibrary{breakable}
% \tcbuselibrary{theorems}
\tcbuselibrary{skins}
\usepackage{microtype}
\usepackage{nicematrix}
\NiceMatrixOptions{cell-space-limits = 5pt}

\renewcommand{\thechapter}{\Roman{chapter}}
\counterwithout{section}{chapter}
\counterwithout{figure}{chapter}
\counterwithout{table}{chapter}
\renewcommand*\thesection{\arabic{section}}

\titleformat{\chapter}[display]
{\bfseries\Large}
{\filleft
	\MakeUppercase{\chaptertitlename} \Huge\thechapter}
{3ex}
{\titlerule
	\vspace{2ex}%
	\filright}
[{%
			\vspace{2ex}%
			\titlerule
		}]

\titlecontents{chapter}
[0pt]
{\addvspace{1pc}}%
{\contentsmargin{0pt}%
	\bfseries
	\makebox[0pt][r]{\huge\thecontentslabel\enspace}%
	\large}
{% \addvspace{.2pc}%
	\contentsmargin{0pt}%
	\large}
{}
[\addvspace{.5pc}]

\newcommand{\chaptertoc}{%
	\dotfill
	\vspace*{1ex}
	\startcontents[chapters]
	\printcontents[chapters]{}{1}[2]{}
	\vspace*{1ex}
	\noindent\dotfill\\
	\vspace*{1pc}
}
% \usepackage{fancyhdr}
% \pagestyle{fancy}

\usetikzlibrary{positioning,quotes,calc,patterns}

%% Equation numbered by section
\numberwithin{equation}{section}

%% Theorem environments
\theoremstyle{plain}
\newtheorem{theorem}{Theorem}[section]
\newtheorem{lemma}[theorem]{Lemma}
\newtheorem{proposition}[theorem]{Proposition}
\newtheorem{corollary}[theorem]{Corollary}

\newtheorem{condition}{Condition}
\renewcommand*{\thecondition}{\Alph{condition}}
\crefname{condition}{condition}{conditions}

\newtheorem*{convention}{Convention}
\newtheorem*{theorem*}{Theorem}
\newtheorem*{lemma*}{Lemma}
\newtheorem*{prop*}{Proposition}
\newtheorem*{corollary*}{Corollary}

\theoremstyle{definition}
\newtheorem{definition}[theorem]{Definition}
% \newtheorem{example}[theorem]{Example}
\newtheorem{exercise}[theorem]{Exercise}
\newtheorem*{notation}{Notation}
\newtheorem*{definition*}{Definition}
\newtheorem*{example*}{Example}
\newtheorem*{exercise*}{Exercise}

\theoremstyle{remark}
\newtheorem{remark}[theorem]{Remark}
\newtheorem*{remark*}{Remark}

\tcolorboxenvironment{notation}{%
	blanker,breakable,left=5mm,
	before skip=10pt,after skip=10pt,
	borderline west={1mm}{0pt}{gray}}

\usepackage{thmtools}

\declaretheoremstyle[
  sibling=theorem,
	style=definition,
  spaceabove=1em plus 0.75em minus 0.25em,
  spacebelow=1em plus 0.75em minus 0.25em,
  qed={\itshape End of example.}
]{exmpstyle}

\declaretheorem[
  style=exmpstyle,
  title=Example,
  refname={example,examples},
  Refname={Example,Examples}
]{example}

%===========================================================%
%The code below customises theorem numbering
%===========================================================%

% \theoremstyle{plain}
% \newtheorem{innercustomgenericplain}{\customgenericname}
% \providecommand{\customgenericname}{}
% \newcommand{\newcustomtheoremplain}[2]{%
% 	\crefname{#2}{#2}{#2s}%
% 	\newenvironment{#1}[1]
% 	{%
% 		\renewcommand\customgenericname{#2}%
% 		\crefalias{innercustomgenericplain}{#2}%
% 		\renewcommand\theinnercustomgenericplain{##1}%
% 		\innercustomgenericplain
% 	}
% 	{\endinnercustomgenericplain}
% }

% \theoremstyle{definition}
% \newtheorem{innercustomgenericdefinition}{\customgenericname}
% \providecommand{\customgenericname}{}
% \newcommand{\newcustomtheoremdefinition}[2]{%
% 	\crefname{#2}{#2}{#2s}%
% 	\newenvironment{#1}[1]
% 	{%
% 		\renewcommand\customgenericname{#2}%
% 		\crefalias{innercustomgenericdefinition}{#2}%
% 		\renewcommand\theinnercustomgenericdefinition{##1}%
% 		\innercustomgenericdefinition
% 	}
% 	{\endinnercustomgenericdefinition}
% }

% \theoremstyle{remark}
% \newtheorem{innercustomgenericremark}{\customgenericname}
% \providecommand{\customgenericname}{}
% \newcommand{\newcustomtheoremremark}[2]{%
% 	\crefname{#2}{#2}{#2s}%
% 	\newenvironment{#1}[1]
% 	{%
% 		\renewcommand\customgenericname{#2}%
% 		\crefalias{innercustomgenericremark}{#2}%
% 		\renewcommand\theinnercustomgenericremark{##1}%
% 		\innercustomgenericremark
% 	}
% 	{\endinnercustomgenericremark}
% }

% \newcustomtheoremplain{customtheorem}{Theorem}
% \newcustomtheoremplain{customlemma}{Lemma}
% \newcustomtheoremplain{customprop}{Proposition}
% \newcustomtheoremplain{customcorollary}{Corollary}

% \newcustomtheoremdefinition{customdefinition}{Definition}
% \newcustomtheoremdefinition{customexample}{Example}
% \newcustomtheoremdefinition{customexercise}{Exercise}

% \newcustomtheoremremark{customremark}{Remark}
