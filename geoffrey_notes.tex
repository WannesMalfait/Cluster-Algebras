\documentclass{article}

%% PACKAGES

% Fancy chapters with toc
\usepackage{titlesec,titletoc}

\usepackage[style=alphabetic, maxnames=4]{biblatex}
\addbibresource{references.bib}

\usepackage{amsmath,amsfonts,amsthm,amssymb,mathtools}
\usepackage[shortlabels]{enumitem}
\usepackage{imakeidx}
\makeindex[intoc] % Add index to bibliography.
\usepackage[bookmarksdepth=2]{hyperref}
\usepackage{xcolor}
\hypersetup{
	colorlinks=true,
	linkcolor={red!50!black},
	citecolor={blue!50!black},
	urlcolor={blue!80!black}
}
\usepackage{cleveref}
\usepackage{tikz-cd}
\usepackage{aligned-overset}
\usepackage{tcolorbox}
\tcbuselibrary{breakable}
% \tcbuselibrary{theorems}
\tcbuselibrary{skins}
\usepackage{microtype}
\usepackage{nicematrix}
\NiceMatrixOptions{cell-space-limits = 5pt}

\renewcommand{\thechapter}{\Roman{chapter}}
\counterwithout{section}{chapter}
\counterwithout{figure}{chapter}
\counterwithout{table}{chapter}
\renewcommand*\thesection{\arabic{section}}

\titleformat{\chapter}[display]
{\bfseries\Large}
{\filleft
	\MakeUppercase{\chaptertitlename} \Huge\thechapter}
{3ex}
{\titlerule
	\vspace{2ex}%
	\filright}
[{%
			\vspace{2ex}%
			\titlerule
		}]

\titlecontents{chapter}
[0pt]
{\addvspace{1pc}}%
{\contentsmargin{0pt}%
	\bfseries
	\makebox[0pt][r]{\huge\thecontentslabel\enspace}%
	\large}
{% \addvspace{.2pc}%
	\contentsmargin{0pt}%
	\large}
{}
[\addvspace{.5pc}]

\newcommand{\chaptertoc}{%
	\dotfill
	\vspace*{1ex}
	\startcontents[chapters]
	\printcontents[chapters]{}{1}[2]{}
	\vspace*{1ex}
	\noindent\dotfill\\
	\vspace*{1pc}
}
% \usepackage{fancyhdr}
% \pagestyle{fancy}

\usetikzlibrary{positioning,quotes,calc,patterns}

%% Equation numbered by section
\numberwithin{equation}{section}

%% Theorem environments
\theoremstyle{plain}
\newtheorem{theorem}{Theorem}[section]
\newtheorem{lemma}[theorem]{Lemma}
\newtheorem{proposition}[theorem]{Proposition}
\newtheorem{corollary}[theorem]{Corollary}

\newtheorem{condition}{Condition}
\renewcommand*{\thecondition}{\Alph{condition}}
\crefname{condition}{condition}{conditions}

\newtheorem*{convention}{Convention}
\newtheorem*{theorem*}{Theorem}
\newtheorem*{lemma*}{Lemma}
\newtheorem*{prop*}{Proposition}
\newtheorem*{corollary*}{Corollary}

\theoremstyle{definition}
\newtheorem{definition}[theorem]{Definition}
% \newtheorem{example}[theorem]{Example}
\newtheorem{exercise}[theorem]{Exercise}
\newtheorem*{notation}{Notation}
\newtheorem*{definition*}{Definition}
\newtheorem*{example*}{Example}
\newtheorem*{exercise*}{Exercise}

\theoremstyle{remark}
\newtheorem{remark}[theorem]{Remark}
\newtheorem*{remark*}{Remark}

\tcolorboxenvironment{notation}{%
	blanker,breakable,left=5mm,
	before skip=10pt,after skip=10pt,
	borderline west={1mm}{0pt}{gray}}

\usepackage{thmtools}

\declaretheoremstyle[
  sibling=theorem,
	style=definition,
  spaceabove=1em plus 0.75em minus 0.25em,
  spacebelow=1em plus 0.75em minus 0.25em,
  qed={\itshape End of example.}
]{exmpstyle}

\declaretheorem[
  style=exmpstyle,
  title=Example,
  refname={example,examples},
  Refname={Example,Examples}
]{example}

%===========================================================%
%The code below customises theorem numbering
%===========================================================%

% \theoremstyle{plain}
% \newtheorem{innercustomgenericplain}{\customgenericname}
% \providecommand{\customgenericname}{}
% \newcommand{\newcustomtheoremplain}[2]{%
% 	\crefname{#2}{#2}{#2s}%
% 	\newenvironment{#1}[1]
% 	{%
% 		\renewcommand\customgenericname{#2}%
% 		\crefalias{innercustomgenericplain}{#2}%
% 		\renewcommand\theinnercustomgenericplain{##1}%
% 		\innercustomgenericplain
% 	}
% 	{\endinnercustomgenericplain}
% }

% \theoremstyle{definition}
% \newtheorem{innercustomgenericdefinition}{\customgenericname}
% \providecommand{\customgenericname}{}
% \newcommand{\newcustomtheoremdefinition}[2]{%
% 	\crefname{#2}{#2}{#2s}%
% 	\newenvironment{#1}[1]
% 	{%
% 		\renewcommand\customgenericname{#2}%
% 		\crefalias{innercustomgenericdefinition}{#2}%
% 		\renewcommand\theinnercustomgenericdefinition{##1}%
% 		\innercustomgenericdefinition
% 	}
% 	{\endinnercustomgenericdefinition}
% }

% \theoremstyle{remark}
% \newtheorem{innercustomgenericremark}{\customgenericname}
% \providecommand{\customgenericname}{}
% \newcommand{\newcustomtheoremremark}[2]{%
% 	\crefname{#2}{#2}{#2s}%
% 	\newenvironment{#1}[1]
% 	{%
% 		\renewcommand\customgenericname{#2}%
% 		\crefalias{innercustomgenericremark}{#2}%
% 		\renewcommand\theinnercustomgenericremark{##1}%
% 		\innercustomgenericremark
% 	}
% 	{\endinnercustomgenericremark}
% }

% \newcustomtheoremplain{customtheorem}{Theorem}
% \newcustomtheoremplain{customlemma}{Lemma}
% \newcustomtheoremplain{customprop}{Proposition}
% \newcustomtheoremplain{customcorollary}{Corollary}

% \newcustomtheoremdefinition{customdefinition}{Definition}
% \newcustomtheoremdefinition{customexample}{Example}
% \newcustomtheoremdefinition{customexercise}{Exercise}

% \newcustomtheoremremark{customremark}{Remark}

\input{definitions/macros}
\input{definitions/letterfonts}

\usepackage[width=400pt]{geometry}

\DeclareMathOperator{\Gr}{Gr}

\title{Geoffrey's talks on cluster algebras}
\author{Wannes Malfait}

\begin{document}

\maketitle
\newpage
\tableofcontents
\newpage

\section{Introduction}

The goal of these lectures will be to cover the theory of cluster algebras related to
Geoffrey's research. The plan is to do the following in each lecture:
\begin{enumerate}
	\item The definition of cluster algebras with, and without, coefficients. Some basic examples
	      of cluster algebras, and the statement of the classification of ``finite type'' cluster
	      algebras.
	\item Here we cover examples from algebraic geometry, and look at the question:
	      \begin{equation*}
		      \bbC [X] \overset{?}{\cong} \text{ cluster algebra } \mcA,
	      \end{equation*}
	      where $X$ is some geometric object like an affine variety.
	      Examples here are those of Dynkin type.
	      For example, those of type $A_n$ will correspond to the
	      Grassmannians $\Gr(2,n+3)$ which in turn correspond to triangulations of the $(n+3)$-gon.
	\item General definition of coefficients, which leads to definition of cluster algebras of
	      geometric type.
	\item $\otimes$-categories
	\item A word about tropical geometry.
\end{enumerate}

In practice, the actual content will be different.

\section{Lecture 04/10/2023}

Let $Q$ be a \emph{quiver}. That is, $Q$ is a directed graph with possible, multiple
edges between a given pair of vertices. We'll assume that there are no loops or
2-cycles. Let $\mcF$ be the field of fractions over some variables $X_1, \dots, X_n$.
\begin{definition}
	We call a pair $(Q, U)$ a \emph{seed} if
	\begin{enumerate}
		\item $Q$ is a finite quiver with vertices $Q_0 = \{1, \dots, n\}$.
		\item $U = \{u_1, \dots, u_n\}$ free field generators of $\mcF$.
		      These are the \emph{cluster variables} of the cluster $U$.
	\end{enumerate}
\end{definition}

Given a seed We can convert it to another seed through a \emph{mutation}.
\begin{definition}
	Let $k \in Q_0$ be a vertex of the quiver $Q$.
	We define the \emph{mutation in direction $k$} of the seed
	$(Q, U)$ to be the seed $\mu_k(Q, U) = (Q', U')$, where
	\begin{enumerate}
		\item $Q'_0 = Q_0$.
		\item All arrows in $Q$ incident to $k$ get their orientation flipped.
		\item For every pair of vertices $i,j$ with $s$ arrows $i \to k$ and $t$ arrows $k \to j$ we
		      ``collapse'' these arrows into $st$ arrows $i \to j$, cancelling pairwise with any
		      arrows $j \to i$.
		      \begin{equation*}
			      \begin{tikzcd}[column sep=small]
				      i \ar[rr, "r"] \ar[rd, "s"] && j \\
				      & k \ar[ur, "t"]
			      \end{tikzcd}
			      \quad\overset{\mu_k}{\leadsto }\quad
			      \begin{tikzcd}[column sep=small]
				      i \ar[rr, "r + st"] \ar[rd,leftarrow, "s"] && j \\
				      & k \ar[ur, leftarrow, "t"]
			      \end{tikzcd}
		      \end{equation*}
		      Here we take $r$ to be negative, if the arrows actually point the other way.
		\item $U' = \{u_1', \dots, u_n'\}$, where $u_i' = u_i$ for $i \neq k$, and $u'_k$
		      is given by the \emph{exchange relation}
		      \begin{equation*}
			      u'_k = \frac{1}{u_k} \left(\prod_{i \to k} u_i + \prod_{k \to j} u_j\right),
		      \end{equation*}
		      where we take the product equal to 1 if $\{i \to k\} = \emptyset$.
	\end{enumerate}
\end{definition}
\begin{remark}
	$\mu_k$ is an involution! This is clear for the quiver, while for the cluster itself we have
	\begin{align*}
		u_k''
		 & = \frac{1}{u_k'}\left(\prod_{k \to i}u_i' + \prod_{j \to k} u_j'\right)                                                \\
		 & = u_k \left(\prod_{i \to k} u_i + \prod_{k \to j}u_j\right)^{-1} \left(\prod_{k \to i}u_i + \prod_{j \to k} u_j\right) \\
		 & = u_k,
	\end{align*}
	as $u_i' = u_i$ for all $i \neq k$, and the expression in the product
	is symmetric with respect to flipping all arrows incident with $k$.
\end{remark}
Let us look at some examples.
\begin{example}
	We start with
	\begin{equation*}
		\begin{tikzcd}[column sep=small]
			& 1 \ar[dr]\\
			2 \ar[ur] && 3 \ar[ll]
		\end{tikzcd}
		\quad U = \{x_1, x_2, x_3\}.
	\end{equation*}
	After mutation in direction 1, we obtain
	\begin{equation*}
		\begin{tikzcd}[column sep=small]
			& 1 \ar[dr, leftarrow]\\
			2 \ar[ur, leftarrow] && 3
		\end{tikzcd}
		\quad U' = \{\frac{x_2 + x_3}{x_1}, x_2, x_3\}.
	\end{equation*}
	Mutation in direction 2 now gives
	\begin{equation*}
		\begin{tikzcd}[column sep=small]
			& 1 \ar[dr, leftarrow]\\
			2 \ar[ur] && 3
		\end{tikzcd}
		\quad U'' = \{\frac{x_2 + x_3}{x_1}, \frac{x_1 + x_2 + x_3}{x_2 x_1}, x_3\}.
	\end{equation*}
\end{example}

\begin{example}
	We start with the $A_3$ quiver
	\begin{equation*}
		Q =
		\begin{tikzcd}
			1 \rar[] & 2 \rar[] &3
		\end{tikzcd}
		\quad \{x_1, x_2, x_3\}.
	\end{equation*}
	Mutation in direction 2 gives
	\begin{equation*}
		\mu_2(Q) =
		\begin{tikzcd}
			1 \rar[leftarrow] \ar[rr, bend right] & 2 \rar[leftarrow]& 3
		\end{tikzcd}
		\quad \{x_1, \frac{x_1 + x_3}{x_2}, x_3\}.
	\end{equation*}
	If we instead apply a mutation in direction 1, we find
	\begin{equation*}
		\mu_1(Q) =
		\begin{tikzcd}
			1 \rar[leftarrow] & 2 \rar[]& 3
		\end{tikzcd}
		\quad \{\frac{x_2 + 1}{x_1}, x_2, x_3\}.
	\end{equation*}
	Finally, a mutation in direction 3 would give
	\begin{equation*}
		\mu_3(Q) =
		\begin{tikzcd}
			1 \rar[] & 2 \rar[leftarrow]& 3
		\end{tikzcd}
		\quad \{x_1, x_2, \frac{x_2 + 1}{x_3}\}.
	\end{equation*}
	Mutating in direction 1 on $\mu_2(Q)$ gives
	\begin{equation*}
		\mu_1(\mu_2(Q)) =
		\begin{tikzcd}
			1 \rar[] \ar[rr,leftarrow, bend right] & 2 & 3
		\end{tikzcd}
		\quad \{\frac{x_1 + x_3 + x_2x_3}{x_1x_2}, \frac{x_1 + x_3}{x_2}, x_3\}.
	\end{equation*}
	If we now mutate in direction 3, we obtain
	\begin{equation*}
		\mu_3(\mu_1(\mu_2(Q))) =
		\begin{tikzcd}
			1 \rar[] \ar[rr, bend right] & 2 & 3
		\end{tikzcd}
		\quad \{\frac{x_1 + x_3 + x_2x_3}{x_1x_2}, \frac{x_1 + x_3}{x_2}, \frac{x_1 x_2 + x_1 + x_3 + x_2 x_3}{x_1x_2x_3}\}.
	\end{equation*}
	It seems that the expressions keep getting messier and messier. We mutate again in direction 2:
	\begin{equation*}
		\mu_2(\mu_3(\mu_1(\mu_2(Q)))) =
		\begin{tikzcd}
			1 \rar[leftarrow] \ar[rr, bend right] & 2 & 3
		\end{tikzcd},
	\end{equation*}
	\begin{equation*}
		\{\frac{x_1 + x_3 + x_2x_3}{x_1x_2}, \frac{x_2 + 1}{x_1}, \frac{x_1 x_2 + x_1 + x_3 + x_2 x_3}{x_1x_2x_3}\}.
	\end{equation*}
	Somehow, some miraculous cancelation happened.
	Note that the expression $\frac{x_2 + 1}{x_2}$ already appeared previously.
	If we instead mutate in direction 1, we get
	\begin{equation*}
		\mu_1(\mu_3(\mu_1(\mu_2(Q)))) =
		\begin{tikzcd}
			1 \rar[leftarrow] \ar[rr,leftarrow, bend right] & 2 & 3
		\end{tikzcd},
	\end{equation*}
	\begin{equation*}
		\{\frac{x_1 + x_3 + x_1x_2}{x_2x_3}, \frac{x_1 + x_3}{x_2}, \frac{x_1 x_2 + x_1 + x_3 + x_2 x_3}{x_1x_2x_3}\}.
	\end{equation*}
	This still introduces a new expression.
	Interestingly, these are all the possible expressions we can obtain!
	If we now mutate in, say, direction 3, we would get as cluster
	\begin{equation*}
		\{\frac{x_1 + x_3 + x_1x_2}{x_2x_3}, \frac{x_1 + x_3}{x_2}, x_1 \}.
	\end{equation*}
	This is again a dramatic cancelation. One can check manually
	that no mutations introduce new cluster variables.

	Another remark is that the total number of cluster variables we obtained was $9 = 3 +
		6$. It is no coincidence that $A_3$ is a Dynkin diagram of rank 3, and that the
	corresponding root system has $6$ positive roots. The positive roots correspond to the
	denominators that appear in the cluster variables.
\end{example}
There are numerous things we can observe (and conjecture) from the two examples.
Here is a short summary:
\begin{theorem}[Fomin--Zelivinsky]
	\leavevmode
	\begin{enumerate}
		\item All the cluster algebras are Laurent polynomials in the variables of any given cluster.
		      \cite[Theorem 3.1]{FominZelivinsky2002CAF}
		\item If $Q$ is a Dynkin diagram, then there is a bijective correspondence between the
		      cluster variables and the almost positive roots ($Phi_{\geq -1}$) given by
		      \begin{equation*}
			      x[\alpha] = \frac{P_\alpha (U_0)}{x^\alpha},
		      \end{equation*}
		      where $P_\alpha$ is a polynomial and $U_0$ is the initial cluster (\cite[Theorem 1.9]{FominZelivinsky2003CAFin}).
		\item For all unoriented quivers of Dynkin type, all orientations are mutation equivalent. If
		      $Q_1$ and $Q_2$ are of different Dynkin type, then they are not mutation equivalent
		      (\cite[Theorem 1.7]{FominZelivinsky2003CAFin})
		\item The number of cluster variables is finite if and only if $Q$ is of Dynkin type
		      (\cite[Theorem 1.6]{FominZelivinsky2003CAFin}).
	\end{enumerate}
\end{theorem}

\textbf{TODO: Which theorem is it from Caldero and Keller relating to acyclic quivers and trees?}

\section{Lecture 07/11/2023}

In the previous lecture, we talked about seeds as pairs $(Q, U)$ where $Q$ was a
quiver. In this session we will sometimes think of it as a pair $(B, U)$ where $B$ is a
skew-symmetrizable matrix. We say that $B$ is the \emph{exchange matrix}. When $Q$ is a
quiver, $B$ is just the adjacency matrix of $Q$. There are two types of ``finiteness''
that we can associate with a cluster algebra. We will call the cluster algebra,
\emph{cluster-finite} if it has only finitely many cluster variables. When there are
only finitely many quivers or matrices that appear under mutation, it will be called
\emph{mutation finite}.

A \emph{cluster monomial} is a monomial in the variables of one cluster. There is the
following characterization of cluster-finite cluster algebras in terms of cluster
monomials.
\begin{theorem}[Gross--Hacking--Keel--Kontsevich, 2018]
	\textbf{TODO: which exact theorem is this, and what paper?}
	A cluster algebra is cluster-finite if and only if the cluster monomials form a basis.
\end{theorem}

Mutation-finite cluster algebras fall under a few categories:
\begin{enumerate}
	\item Cluster-finite cluster algebras.
	\item Cluster algebras originating from surfaces with boundary and punctures
	\item A few exceptions.
\end{enumerate}
See \cite[Theorem 1.10]{FeliksonPavel2023cluster} for a more precise statement.
An example of a quiver that is mutation-finite but not cluster-finite is the ``Markov quiver''
\begin{equation*}
	\begin{tikzcd}[column sep= small]
		& \bullet \ar[ld, shift left] \ar[ld, shift right]\\
		\bullet \ar[rr, shift left] \ar[rr, shift right] && \bullet \ar[lu, shift left] \ar[lu, shift right]
	\end{tikzcd}.
\end{equation*}
It corresponds to the torus with 1 puncture.
It can also be shown that its cluster algebra is strictly contained in its upper cluster algebra.

\subsection{Coefficients}

Let $\mcF = \bbQ(u_1, \dots, u_n)$. A \emph{labeled seed of geometric type} is a pair
$(\mathbf{x}, \tilde{Q})$ where
\begin{itemize}
	\item $\mathbf{x} = \{x_1, \dots, x_m\}$ is a free generating set of $\mcF$, and
	\item $\tilde{Q} = Q \cup Q_{\text{fr}}$ is a quiver with frozen vertices $Q_{\text{fr}}$.
	      There are no arrows between any of the frozen vertices. And mutations only happen
	      on the non-frozen vertices, the \emph{exchangeable} vertices.
\end{itemize}
In terms of exchange matrices, $Q$ corresponds to a skew symmetric matrix $B$ via
$b_{ij} = \# i \to j - \# j \to i$, which is referred to as the \emph{principal part} of
\begin{equation*}
	\tilde{B} =
	\begin{pmatrix}
		B \\
		\square
	\end{pmatrix}.
\end{equation*}

We will now re-write the exchange relation in a way that can then be generalized:
\begin{align*}
	x_k x_k'
	 & = \prod_{i=1}^m x_i^{[b_{ik}]_+} + \prod_{i = 1}^m x_i^{[-b_{ik}]_+}                                                 \\
	 & = \frac{y_k}{y_k \oplus 1}\prod_{i=1}^n x_i^{[b_{ik}]_+} + \frac{1}{y_k \oplus 1} \prod_{i = 1}^n x_i^{[-b_{ik}]_+}.
\end{align*}
Here we introduced a lot of new notation all at once. We denote $\bbP$ for the tropical semiring,
where
\begin{equation*}
	x_i^{a_i} \oplus x_i^{b_i} = x_i^{\min\{a_i, b_i\}},
\end{equation*}
and hence $1 \oplus x_i^{b_i} = x_i^{[-b_i]_+}$ with $[a]_+ = \max \{a, 0\}$.
The variables $y$ are defined as
\begin{equation*}
	y_j = \prod_{i = n + 1}^m x_i^{b_{ij}},
\end{equation*}
and often referred to as the \emph{coefficients}.
They satisfy the following ``tropical $y$-seed mutation'':
\begin{equation*}
	y'_j =
	\begin{cases}
		y_k                                            & j = k   \\
		y_iy_k^{[b_{kj}]_+} + (y_k \oplus 1)^{-b_{kj}} & j\neq k
	\end{cases}.
\end{equation*}
Notice how this mutation has a very different dynamic to the one for the exchangeable variables.
All the variables, except the $k$-th one, are mutated, but the mutation rule itself is much simpler.
One can show that the $y$-variables don't satisfy a Laurent phenomenon in general.
Instead of defining the coefficients in terms of the frozen variables, as we did now,
one defines more generally a labeled seed to be a triple $(\mathbf{x}, \mathbf{y}, B)$.

\subsection{Affine varieties}

Let $X$ be an affine algebraic variety. We want to determine when the coordinate
algebra, $\bbC[X]$, can be equipped with the structure of a cluster algebra. Suppose
from now on that $X$ is irreducible and rational, which means that the coordinate
algebra is a domain and purely transcendental. For example, if $X$ contains an open
subvariety that is isomorphic to $\bbA^n$, then $X$ is rational.
\begin{lemma}
	\textbf{TODO: what is the precise statement?
		Do you need the irreducibility and rationality conditions on $X$ for this lemma?}
	If $\phi \colon \mcA \to \bbC[X]$ is an algebra morphism such that the images
	of the cluster variables generate $\bbC[X]$, and satisfy the exchange relations, then
	$phi$ is actually an isomorphism.
\end{lemma}

We will now explore a specific example.
\begin{example}
	Let $V = \bbC^{2k}$ with $k\geq 4$. Let $Q$ be the non-degenerate quadratic form given by
	\begin{equation*}
		Q(x_1, \dots, x_{2k}) = \sum_{i=1}^k (-1)^{i-1}x_i x_{2k+1-i}.
	\end{equation*}
	The zero locus gives a smooth quadric, with coordinate ring
	\begin{equation*}
		A_Q = \bbC[x_1 , \dots, x_{2k}]/Q(x_1, dots, x_n).
	\end{equation*}
	We will now try to put a cluster algebra structure on $A_Q$.
	The construction will be very ad hoc.

	Define
	\begin{equation*}
		p_s = \sum_{i=1}^{j+1}(-1)^{s+1+i}x_i x_{2k+1-i},
	\end{equation*}
	for $1\leq s \leq k-3$. Then, for $2 \leq i \leq k -1$ we have
	\begin{equation*}
		x_i x_{2k+1 - i} =
		\begin{cases}
			p_1 + x_1 - x_{2k}    & i = 2              \\
			p_{i-1} + p_{i-2}     & 3 \leq i \leq k -2 \\
			x_k x_{k+1} + p_{k-3} & i = k - 1
		\end{cases}.
	\end{equation*}
	These are starting to look like exchange relations! We now want
	a cluster algebra with cluster variables $\{x_1 , dots, x_{k-1}\} \cup \{x_{k+2}, \dots, x_{2k-1}\}$, and
	frozen variables $\{x_1, x_{k},x_{k+1}, x_{2k}\} \cup \{p_s\}$. This means that the clusters will contain
	for each $2\leq i \leq k -1$ either $x_i$ or $x_{2k+1-i}$. The details are hard to work out,
	so let us look at a specific case: $k = 5$.
	\begin{equation*}
		\begin{tikzcd}
			& x_2 \ar[rd] & & x_3 \ar[d] & & x_4 \ar[ld] \ar[d] &\\
			\boxed{x_1} \ar[ru] & \boxed{x_{10}} \ar[u] & \boxed{P_1} \ar[ru] & \boxed{P_2} \ar[rru] & \boxed{x_5} & \boxed{x_6}
		\end{tikzcd}.
	\end{equation*}
	Since the exchangeable variables are all disjoint,
	we end up with something of the form $A_1^3$. In general, we end
	up with a cluster structure of type $A_1^{k - 2}$.
\end{example}

\begin{remark}
	The quadric from the example above is a (partial) flag variety.
	\textbf{TODO: How is it a flag variety?}
	This is
	a variety whose points are flags $V_1 \subseteq \dots \subseteq V_k = V$ in $V$
	of a given signature $(d_1, \dots, d_k)$ i.e., $\dim(V_1) = d_1, \dots, \dim(V_k) = d_k$.
	This is isomorphic to the homogeneous space $\SL(V)/P$ where $P$ is the parabolic
	subgroup given by the stabilizer of the flags. When $k=2$, a partial flag variety
	is just the Grassmannian $\Gr(d_1, V)$.
\end{remark}

In a more general context, Gei{\ss} Leclerc and Schr\"oer showed the existence of
cluster algebra structures on partial flag varieties coming from semisimple Lie groups.
This was done through preprojective algebras \cite{GeissLeclercSchroer2008PFvar}.
\textbf{TODO: I really couldn't figure out what was happening from the paper? What did
	they do exactly?}

\bibliographystyle{plain}
\bibliography{references.bib}

\end{document}